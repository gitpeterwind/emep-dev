%%%%%%%%%%%%%%%%%%%%%%%%%%%%%%%%%%%%%%%%%%%%%%%%%%%%%%%%%%%%%%%%%%%%%%%%%%%%%%
\section{sites and sondes}
\label{Output:ascii}

Two main options are available for the output of ascii files for comparison
with measurements or detailed model analysis. These are

\begin{description}
\item[sites]  

      output of surface concentrations for a set of specified
      measurement site locations.
\item[sondes] 

      output of concentrations for the vertical column above
     a set of specified locations.
\end{description}

Both sites and sondes are specified and handled in similar ways, in
the module {\bf Sites\_ml}, so we treat them both together below.

Locations are specified in input files, sites.dat and sondes.dat, whose
directory-locations should be specified in {\bf run.pl}. For example,
a sites.dat file might look like:

\begin{small}\begin{verbatim}
schauinsland    101   51
thessaloniki    132   58
ispra           106   48      Comment: 45.8,8.6333-> 105.864 47.742  (to150old)
testsitemid     120   70      ... just for testing
testsite12      102   88      ... just for testing
border_72_84     72   84      ... just for testing
\end{verbatim}
\end{small}


These file are optional.

The species and meteorological data required are specified in {\bf My\_Outputs}
through the use of arrays. Only a few met fields are defined so far but
more can be added into {\bf Sites\_ml} as required. The outputs consist
of a header giving the number of sites used, species/met data used, and
then actual values specified with a 5es10.3 format. For the sonde data
values are given for all 20 levels, starting with the ground-level values.

