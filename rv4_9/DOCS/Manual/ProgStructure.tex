\chapter{The Code}


\begin{itemize}
\item
95\% Fortran 90/95

\item
Preferably in F
   \begin{itemize}
      \item
        A sub-set of F95
   \end{itemize}

\item
   Heavy use of 'modern' fortran and modules
   \begin{itemize}
      \item
        e.g. No common blocks
      \item
        intent attributes  in subroutine arguments
   \end{itemize}
\end{itemize}

\newpage
\section*{A bit about coding}

All the physical constants like Avogadro's number, molecular weight of
air, Gas constants etc., are defined in the module called
{\bf 'Physicalconstants\_ml}'. A piece of the code is given below. \\
 
\begin{verbatim}
      module PhysicalConstants_ml
!----------------------------------------------------------------------------
!  Defines Physical constants
!----------------------------------------------------------------------------
implicit none
private

  real , public, parameter ::         &
    AVOG   = 6.023e23                 & ! Avogadros number
  , ATWAIR = 28.964                   & ! mol wt of air, g/mol
  , RGAS_ATML = 0.08205               & ! Molar Gas constant (atm M-1 K-1)
  , RGAS_KG   = 287.0                 & ! Molar Gas constant (J K-1 kg-1)
  , RGAS_J    = 8.314                   ! Molar Gas constant (J mol-1 K-1)

  real, public, parameter  ::    &
       GRAV    = 9.807           &   ! Gravity, m s-2
  ....
    ,  FREEPATH  = 6.5e-8        &   ! Mean Free Path of air [m]
  ....

\end{verbatim}

 If you want to use any of these constants
outside this module, one do not need to define it again, instead use
it as given below. 

\begin{verbatim}

module Demo1
  use PhysicalConstants_ml, only : AVOG
 ...
end module Demo1

\end{verbatim}



\section*{A simple module}

\begin{verbatim}
      module PhysicalConstants_ml
!----------------------------------------------------------------------------
!  Defines Physical constants
!----------------------------------------------------------------------------
implicit none
private

  real , public, parameter ::         &
    AVOG   = 6.023e23                 & ! Avogadros number
  , ATWAIR = 28.964                   & ! mol wt of air, g/mol
  , RGAS_ATML = 0.08205               & ! Molar Gas constant (atm M-1 K-1)
  , RGAS_KG   = 287.0                 & ! Molar Gas constant (J K-1 kg-1)
  , RGAS_J    = 8.314                   ! Molar Gas constant (J mol-1 K-1)

  real, public, parameter  ::    &
       GRAV    = 9.807           &   ! Gravity, m s-2
  ....
    ,  FREEPATH  = 6.5e-8        &   ! Mean Free Path of air [m]
  ....

\end{verbatim}



%\newpage
To use, e.g.  AVOG from another module, one usually uses:

\begin{verbatim}

module Demo1
  use PhysicalConstants_ml, only : AVOG
 ...
end module Demo1

\end{verbatim}

More complex modules mix subroutines and data.


A typical EMEP code module looks like
\begin{verbatim}

module Demo2
  use PhysicalConstants_ml, only : AVOG
  use Rsurface_ml,  only : Rsurface, Gsto, Rb
 ...

  !/ Subroutines:

  public :: demo_sub
  private :: Initialise
  private :: Tabulate


  logical, save, private :: my_first_call = .true.

 contains
   subroutine demo_sub(a,b,c)
     real, intent(in) :: a, b
     real, intent(out) :: c

     if ( my_first_call ) then
           call Initialise
           call Tabulate
           my_first_call = .false.
     end if

     call Rsurface(....)

   end subroutine demo_sun

   subroutine Initialise
      bla bla 
   end subroutine Initialise

   subroutine Tabulate
      bla bla 
   end subroutine Tabulate

end module Demo1

\end{verbatim}
