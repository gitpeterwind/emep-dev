\chapter{Submitting a Run}
\label{ch:SubmitARun}

In this chapter we provide detailed information on how to run the
regional Unified EMEP model for two different types of
simulations, namely: 

\begin{itemize}

\item {\bf Base run} \\
This is the default set up for yearly transport model calculations
in 50x50 km$^2$ grid. 
\item{\bf Scenario run} \\
 A run with reduced emissions from a particular country or several
 countries is called 
a ``Scenario run''. It is the basic type of run for source-receptor
calculations. 

\end{itemize}

\noindent
Details about the submission of these
different types of runs are given below. We suggest that users test
the ``Base run'' first, which can be done without significant changes in
the code itself. One can also use the outputs of such a run in the
future as a reference run for the other simulations.\\  
% 
% As explained in the previous chapters, once the model tar file is
% untarred and all the files are located in the directory called
% {\sl Unify/Unimod.rv3\_7/}, one can find the run script with name
% ``modrun.sh'' in this 
% directory.

\newpage
\section{Base run}

This is an example of a minimum modrun.sh script to run the model.


\begin{verbatim}
#!/bin/bash

#Minimum script to run the emep model

#link to the input data
inputdir=./Base
ln -s $inputdir/* .

#define some input data
trendyear=2008 #emission year
runlabel1=rv3_7 #short label
runlabel2=Opensource_setup #long label
startyear=2008 #start year (metdata)
startmonth=1 #start month (metdata)
startday=1 #start day (metdata)
endmonth=1 #end month (metdata)
endday=31 #end day (metdata)

#put input data into a temporary file called INPUT.PARA
cat>>    'INPUT.PARA'<<    EOF
$trendyear
$runlabel1
$runlabel2
$startyear
$startmonth
$startday
$endmonth
$endday
EOF

#run the model
mpirun Unimod

#clean the links to the input data
ls $inputdir|xargs rm
rm INPUT.PARA

\end{verbatim}

The script is pretty self-explanatory, first the script links 
all the input files so you check if all the input files are in 
the right directory. 
The trendyear can be set to change the boundary emissions for 
earlier and future years, see the modules {\bf BoundaryConditions\_ml.f90 } 
and {\bf GlobalBCs\_ml.f90 } to understand better what the trendyear 
setting does. 
The runlabel1 option sets the name of the different output netCDF 
files. 
Startyear has to be set equal to the year 
that the metdata is available. The period you want to run the 
EMEP model is set with startmonth, and startday and the last day is 
set with endmonth and endday. \\

It is recommended to submit the script as a batch job. Please check the submission soutine 
on the computer system you are running on. 
When setting the number of nodes, it is important that it equals to the number set in the fortran module 
{\bf ModelConstants\_ml.f90}. In the module you set the wanted nodes in the x- and y-direction with NPROCX and 
NPROCY. The sum of these two is the number of nodes used. \\

When the job is no longer model is finnished or has crashed for some reason, there will be an output log file and 
an error file in the directory. If the run was successful, a message stating 
\begin{verbatim}
Eulmod: Successful exit atCET ..date..
\end{verbatim}
will be stated at the end of the log file, before listing up site and sonde files. 
The model results will also produced in the same directory, or you can set you own working directory in 
the script. For more information about the model results output files, see chapter \ref{ch:output}.\\

If for some reason 
the model crashed, please check both the log and error file for any clue of the crash. After fixing the 
problem the job can be submitted again. If the model has crashed, then the links to the inputdatas are not cleaned. 

The script can also be submitted interactivelly, and either have the output be written to the screen or to 
named output log and error files. 
 


\subsection{Setting the model domain}

It is possible to run the model on a smaller domain than the full
regional model domain, which is defined by  x coordinates ranging
from 1 to 132 and y coordinates ranging from 1 to 159. 

To set a smaller domain, one needs only to specify the
coordinates of the new domain in {\bf RUNDOMAIN} in the Fortran module
called {\bf ModelConstants\_ml.f90}. For example:


\begin{verbatim}
  
   !                    x0   x1  y0   y1
        RUNDOMAIN = (/  36, 100, 50, 150 /)  ! smaller EECCA domain

\end{verbatim}

tells the model to run in the domain with x coordinates ranging from
36 to 100 and y coordinates from 50 to 150.\\ 

\section{Scenario run}

The EMEP  model can be used to test the impact of reduced emission of
one or more pollutants from a particular country or a number of
countries.  Such runs are called ``Scenario runs''. They are the basic
runs for source-receptor calculations.


Emission factors for reduced emissions of pollutants from different
sectors and countries can be defined in the input file called
``femis.dat'', which can be found in the downloaded input data
directory, {\sl input\_data/Common/}.

An example of the ``femis.dat'' file for a base run is shown below:

\begin{verbatim}
------------------------------------------
Name  5     sox    nox   voc   nh3  pm25
27    0     1.0    1.0   1.0   1.0  1.0  

------------------------------------------
\end{verbatim}

\noindent
This base run example means that there is 100\% (1.0) emission of sox (SO$_x$),
nox (NO$_x$), voc (VOC), nh3 (NH$_3$) and pm25 (PM$_{2.5}$) from all
sectors in the UK. 

\begin{itemize}

\item The first column of
the second line represents the country code. (27 is the code for UK.)
The codes for all countries can be found in  Fortran module {\bf
  Country\_ml.f90}. Please note that the country code must be the same
which is used in the emission files for the given country. Some
countries and areas are divided into sub-areas in the emission
files. In this case, one line for each sub-area has to be included
into the ``femis.dat'' file. Countries and areas where emissions are
given for sub-areas include the Russian Federation, Germany and all 
sea areas.    

\item The second
column of the second line 
represents the sector and ``0'' means all sectors. Here one can write
the appropriate sector code if the emission is reduced only from a specific
sector. The description of each sector can also be found in the Fortran
module {\bf EmisDef\_ml.f90}. 

\item The columns under the pollutant names show the emission factors
  for the given pollutants. For example, 0.7 would mean 70\% of the
  original emission, thus 30\% reduction.


\item The number (``5'') following
the first text (``Name'') in the first line gives the number of
pollutants treated in the file.   
              
\end{itemize}

An example of ``femis.dat'' file describing 50\% reduced emission of
``sox'' from sector 10 (the emission from agriculture) in the UK can be given as:  

\begin{verbatim}
-----------------------------------------
Name  5    sox    nox   voc   nh3  pm25 
27    10   0.5    1.0   1.0   1.0  1.0   

-----------------------------------------
\end{verbatim}        

For a scenario run ``femis.dat'' file should be edited 
manually depending on the factor of
reduction one would like to test with any pollutant from any sector
and/or any country. Several lines can be written in the file.

Once the ``femis.dat''  file is
edited, the run can be submitted.  



