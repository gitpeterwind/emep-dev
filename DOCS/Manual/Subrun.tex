\chapter{Submitting a Run}
\label{ch:SubmitARun}

In this chapter we provide detailed information on how to run the
regional EMEP/MSC-W model for two different types of simulations, namely: 

\begin{description}
\item[Base run]
This is the default set up for yearly transport model calculations
in $50\times50$~km$^2$ grid. 
\item[Scenario run]
 A run with reduced emissions from a particular country or several
 countries is called 
a ``Scenario run''. It is the basic type of run for the source-receptor
calculations. 
\end{description}

\noindent
Details about the submission of these
different types of runs are given below. We suggest that users test
the ``Base run'' first, which can be done without significant changes in
the code itself. One can also use the outputs of such a run in the
future as a reference run for the other simulations.\\  
% 
% As explained in the previous chapters, once the model tar file is
% untarred and all the files are located in the directory called
% {\sl Unify/Unimod.rv3\_7/}, one can find the run script with name
% ``modrun.sh'' in this 
% directory.

\newpage
\section{Base run}

This is an example of a minimum modrun.sh script to run the model.

\begin{quote}
\begin{verbatim}
#!/bin/bash

# Minimalistic script for run the EMEP/MSC-W model

# Link the input data
inputdir=.
ln -s $inputdir/met/*   .   # Driving meteorology
ln -s $inputdir/input/* .   # Other input files

# Define some run parameters
trendyear=2010              # emission year
runlabel1=Base              # short label
runlabel2=Opensource_setup  # long label
startdate="2010 01 01"      # start date (metdata)
  enddate="2010 01 01"      # end date (metdata)

# Put the run parameters in a temporary file
cat > INPUT.PARA << EOF
$trendyear
$runlabel1
$runlabel2
$startdate
$enddate
EOF

# Run the model
mpirun $inputdir/code/Unimod

# Clean the links to the input data and remove INPUT.PARA
ls $inputdir/met  |xargs rm
ls $inputdir/input|xargs rm
rm INPUT.PARA

\end{verbatim}
\end{quote}
This bash shell script is designed so that users can easily 
adapt it to fit their needs. It contain the minimum information 
required to run the EMEP/MSC-W model. 
The script should be self-explanatory. It assumes two directories for input data. 
One with all the metdata for 2010, and for January 1$^{st}$ 2011 (365 +1 files), 
and one input directory for all the other input files. You need to set the right paths 
for the input directories.  
All the input files in the input directories are linked to the directory you are working from. 
  

The next part of the script sets the different parameters for the model run. 
The trendyear can be set to change the boundary emissions for 
earlier and future years, see the modules {\bf BoundaryConditions\_ml.f90 } 
and {\bf GlobalBCs\_ml.f90 } to understand better what the trendyear 
setting does. The default setting is the meteorological year you are running for, 
in this case 2010. 
The runlabel1 option sets the name of the different output netCDF 
files , see Table \ref{ch:output}. 
The startdate and enddate parameters are set for the timeperiod you want the model 
to run (YYYY MM DD), and you need meteorology data for the period.

To run the model, the correct path to the EMEP/MSC-W model code 
has to be set (mpirun path\_to\_the\_modelcode/Unimod).  

It is recommended to submit the script as a batch job. Please check the 
submission routine on the computer system you are running on. 
In the newer model versions (since 4.0) the number of nodes is set 
automatically from what is asked for when submitting a job. 
% When setting the number of nodes, it is important that it equals to the number set in the FORTRAN module 
% {\bf ModelConstants\_ml.f90}. In the module you set the wanted nodes in the x- and y-direction with {\bf NPROCX } and 
% {\bf NPROCY.} The product of these two is the number of nodes used. 
The approximate time and CPU usage is described in Section \ref{sec:compinf}\\

When the job is no longer running or in the queue, it is either finished or has crashed for some reason.  
If the model run crashed, an error message will give information on what was missing or wrong in the routine.  
If the run was successful, a message
\begin{quote}
\begin{verbatim}
 ++++++++++++++++++++++++++++++++++++++++++++++++
 programme is finished
\end{verbatim}
\end{quote}
will be stated at the end of the log file, before printing the Timing.out file.   
The model results will be written to this same directory. Please make 
sure there is enough disk place for the model results. 
%Here again, the user can edit the script to create an output directory. 
For more information about the model result output files, see chapter \ref{ch:output}.\\

If for some reason the model crashed, please check both the log and the error 
file for any clue of the crash. After fixing the problem the job can be 
submitted again. If the model has crashed, then the links to the input data 
are not removed. 

The script can also be submitted interactively, and either have the output 
written to the screen or to named error and output log files. 
 
% \newpage

\subsection{Model constants}

In the module {\bf ModelConstants\_ml.f90 } there is a possibility to set 
different constants and flags when running the model. 
Some of these are briefly explained in Chapter \ref{ch:InputFiles}. 
An example of one of these settings is shown below:

\begin{small}
\begin{quote}
  \begin{verbatim}
logical, public, parameter :: USE\_CONVECTION = .false. 
  \end{verbatim}
\end{quote}
\end{small}

Since convection has a smaller influence on transport on higher latitudes, 
this flag is turned off by default when running the EECCA domain. It is 
however important when running the model globally. A global version 
of the EMEP/MSC-W model will be released later. 

% The module {\bf ModelConstants\_ml.f90 } also sets the number of processors used in each direction with 
% {\bf PROCX } and {\bf NPROCY }. As described above, it is important that the product of these match the number set 
% when submitting the job. 

\subsection{config\_emep.nml}

The model has a namelist system.  Different constants and flags are added in 
this file.  Model gets information about running for special cases from this 
file.   For example, this namelist sets which domain you want to run the model 
on, other than the default domain. 
The datasets provided are for the EMEP grid EECCA. 
It is possible to run the model on a smaller domain than the full
regional model domain, as defined by  x coordinates ranging
from 1 to 132 and y coordinates ranging from 1 to 159. 

To set a smaller domain, one needs only to specify the
coordinates of the new domain in {\bf RUNDOMAIN} in the namelist file,  
 ``config\_emep.nml''. For example:

\begin{quote}
\begin{verbatim}
   !                    x0   x1  y0   y1
        RUNDOMAIN = (/  36, 100, 50, 150 /)  ! smaller EECCA domain
\end{verbatim}
\end{quote}

tells the model to run in the domain with x coordinates ranging from
36 to 100 and y coordinates from 50 to 150.\\ 

% \newpage

\section{Scenario run}
\label{sec:scenrun}

The EMEP/MSC-W model can be used to test the impact of reduced emission of
one or more pollutants from a particular country or a number of
countries.  Such runs are called ``Scenario runs''. They are the basic
runs for source-receptor calculations.

Emission factors for reduced emissions of pollutants from different
sectors and countries can be defined in the input file called
``femis.dat'', which can be found in the downloaded input data
directory, see section \ref{sec:femis}.

An example of the ``femis.dat'' file for a base run is shown below:
\begin{quote}

\begin{verbatim}
------------------------------------------
Name  5     sox    nox   voc   nh3  pm25
27    0     1.0    1.0   1.0   1.0  1.0  

------------------------------------------
\end{verbatim}
\end{quote}
\noindent
This base run example means that there are (1.0), no emission reductions of 
sox (SO$_x$), nox (NO$_x$), voc (VOC), nh3 (NH$_3$) and pm25 (PM$_{2.5}$) from 
all sectors in the UK. 

\begin{itemize}

\item The first column of
the second line represents the country code. (27 is the code for UK.)
The codes for all countries can be found in  Fortran module {\bf
  Country\_ml.f90}. Please note that the country code must be the same
as in the emission files for the given country. Some
countries and areas are divided into sub-areas in the emission
files. In this case, one line for each sub-area has to be included
into the ``femis.dat'' file. Countries and areas where emissions are
given for sub-areas include the Russian Federation, Germany and all 
sea areas.    

\item The second
column of the second line 
represents the sector and ``0'' means all sectors. Here one can write
the appropriate sector code if the emission is reduced only from a specific
sector. The description of each sector can also be found in the Fortran
module {\bf EmisDef\_ml.f90}. 

\item The columns under the pollutant names show the emission factors
  for the given pollutants. For example, 0.7 would mean 70\% of the
  original emission, thus 30\% reduction.

\item The number (``5'') following
the first text (``Name'') in the first line gives the number of
pollutants treated in the file.   
              
\end{itemize}


An example of ``femis.dat'' file describing 50\% reduced emission of
``sox'' from sector 10 (the emission from agriculture) in the UK can be given as:  
\begin{quote}
\begin{verbatim}
-----------------------------------------
Name  5    sox    nox   voc   nh3  pm25 
27    10   0.5    1.0   1.0   1.0  1.0   

-----------------------------------------
\end{verbatim}        
\end{quote}
For a scenario run ``femis.dat'' file should be edited 
manually depending on the level of
reduction one would like to test with any pollutant from any sector
and/or any country. Several lines can be written in the file.
% Once the ``femis.dat''  file is
% edited, the scenario run can be submitted.  
