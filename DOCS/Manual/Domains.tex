\chapter{Domains and Coordinates}

\section{Domains}

Domains - a confusing word. In normal modelling terminology domains
refer to the total area covered by the model simulation, and possibly
to some output areas. In the unified model, and parallel-coding terminology,
domains refer also to the area covered by each processor. To try to distinguish
between these, we generally refer to the former type as `full' domains, 
and the latter as local domains.  At present the `full' domain in the
EMEP model is defined by the meteorology. Thus, for the traditional EMEP
grid this full domain extends 170 grid squares in the x-direction, and
133 grid squares in the y-direction.

The model has a number of domains, illustrated in Fig.\ref{fig:Domains}

\begin{figure}
\begin{picture}(400,300)
\put(2,2){\framebox(340,266)[br]{Meteorology}}
\put(72,20){\framebox(260,240)[br]{EMEP area}}
\put(92,40){\framebox(200,160)[br]{Run area}}
\put(130,60){\dashbox{0.9}(180,180)[br]{Hourly output}}
\put(200,90){\dashbox{0.6}(30,30)[br]{Restri output}}
\end{picture}
\caption{Illustrations of different domains in the EMEP
model.}
\label{fig:Domains}
\end{figure}



\subsection{Full (Meteorology) area} 

Specifies the master coordinates to be used in the model. 
(As noted above, at present the `full' domain and
 meteorology domain are identical).


\noindent
Set in {\bf run.pl} : @largedomain = ( 1 , 171, 1 , 133 );\\
\vspace{3mm}

Translated into \Par\ as:\\
\begin{small}\begin{verbatim}
  integer, public, parameter ::  &
    IILARDOM    =    170         &
  , JJLARDOM    =    133         &
\end{verbatim}
\end{small}



Translated as for Run area (below)

\section{ Run area  (and its shadow!) } 

\noindent
Set in {\bf run.pl} : @smalldomain = ( user-specified  );  \\
\vspace{3mm}
\\
For example, @smalldomain = ( 101, 140, 51, 90 ) 
translates to the following in \Par:\\
\begin{small}\begin{verbatim}
  integer, public, parameter ::  &
  , ISMBEG      =   101    & ! Starting x-coodinate
  , JSMBEG      =   51     & ! Starting y-coodinate
  , GIMAX       =   40     & ! Number of global points in longitude
  , GJMAX       =   40     & ! Number of global points in latitude
\end{verbatim}
\end{small}

The traditional EMEP run area corresponds to  @rundomain = ( 36, 160, 11, 123 ).


The area run by the model does not need to be the EMEP area, just
any sub-domain of the meteorology domain. For testing we could choose just to
have a tiny run-domain. For testing with 4 or 8 processors then
\@smalldomain = ( 51, 160, 31, 120 ) works quite well.



\subsection*{Shadow area}

COULD MOVE THE FOLLOWING TO A PART B, FOR CODING. ONLY REALLY RELEVENT
FOR USERS WHO WANT TO DIG  INTO THE CODE - EXCEPT TO NOTE THAT THERE
IS AN `EDGE' FOR WHICH CALULATIONS AREN'T DONE.

IMPORTANT when coding - the grid cells on the boundaries of the run-area are
excluded from the chemical calculations, since the advection routines
require one cell more than the chemistry. Thus, in \Par\ you will find:

\begin{small}\begin{verbatim}
From par_ml.f90
     limax       & ! Actual number of local points in longitude
   , ljmax         ! Actual number of local points in latitude

     li0  & ! First local index in longitude when outer boundary is excluded
   , li1  & ! Last local index in longitude when outer boundary is excluded
   , lj0  & ! First local index in latitude when outer boundary is excluded
   , lj1    ! Last local index in latitude when outer boundary is excluded
\end{verbatim}
\end{small}


So, if you loop over something that does not need
info from neighboring boxes then it is ok to use li0:li1, lj0:lj1,
but if you need info from the "shadow area" (ie advection)  you must use
1:limax,1:ljmax.
\bigskip
\bigskip


The domains used for the outputs are discussed more in section~\ref{OUTPUTS}.



\subsection{Summary of coordinates}


Table~\ref{Tab:COORDS}
gives an example of the various coordinates and limits for the case of
4 processors (me=0,1,2,3), and a run domain set with:

\begin{verbatim}
@rundomain = (  76, 117,  40, 102 ) ; 
\end{verbatim}

NEED TO EXPLAIN i\_glob, j\_glob, WHY MAXLIMAX+1.

\begin{table}
\caption{Example of coordinates derived for each processor with
the case of 4 processors (me=0,1,2,3) and smalldomain:  76, 117,  40, 102}
For example, processor me=0 covers ..... continue


\label{Tab:COORDS}
\begin{tabular}{lcccc}\\
me& 0& 1& 2& 4\\
ISMBEG& 76& 76& 76& 76\\
JSMBEG& 40& 40& 40& 40\\
gi0& 1& 22& 1& 22\\
gj0& 1& 1& 32& 32\\
li0& 2& 1& 2& 1\\
li1& 21& 20& 21& 20\\
limax& 21& 21& 21& 21\\
MAXLIMAX& 21& 21& 21& 21\\
lj0& 2& 2& 1& 1\\
lj1& 31& 31& 31& 31\\
MAXLJMAX& 32& 32& 32& 32\\
i\_glob(1)& 76& 97& 76& 97\\
i\_glob(MAXLIMAX+1)& 97& 118& 97& 118\\
j\_glob(1)& 40& 40& 71& 71\\
j\_glob(MAXLJMAX+1)& 72& 72& 103& 103\\
\end{tabular}
\end{table}
