\chapter{Domains}

Domains - a confusing word. In normal modelling terminology domains
refer to the total area covered by the model simulation, and possibly
to some output areas. In the unified model, and parallel-coding terminology,
domains refer also to the area covered by each processor. To try to distinguish
between these, we generally refer to the former type as global domains, 
and the latter as local domains. (The word global is also confusing of course - here we 
do not mean for the whole globe!)

The model has a number of global-type domains:\\
\\
\begin{picture}(400,300)
\put(2,2){\framebox(340,266)[br]{Meteorology}}
\put(72,20){\framebox(260,240)[br]{EMEP area}}
\put(92,40){\framebox(200,160)[br]{Run area}}
\put(130,60){\dashbox{0.9}(180,180)[br]{Hourly output}}
\put(200,90){\dashbox{0.6}(30,30)[br]{Restri output}}
\end{picture}


%TMP\section{Meteorology}
%TMP
%TMP\noindent
%TMPSet in {\bf run.pl} : @largedomain = ( 1 , 171, 1 , 133 );\\
%TMP\vspace{3mm}
%TMP
%TMPTranslated into \Par\ as:\\
%TMP\begin{small}\begin{verbatim}
%TMP  integer, public, parameter ::  &
%TMP    IILARDOM    =    170         &
%TMP  , JJLARDOM    =    133         &
%TMP\end{verbatim}
%TMP\end{small}
%TMP
%TMP
%TMP\section{EMEP area} 
%TMP
%TMP\noindent
%TMPSet in {\bf run.pl} : @smalldomain = ( 36, 160, 11, 123 ) ;
%TMP\vspace{3mm}
%TMP
%TMPTranslated as for Run area (below)
%TMP
%TMP\section{ Run area  (and its shadow!) } 
%TMP
%TMP\noindent
%TMPSet in {\bf run.pl} : @smalldomain = ( user-specified  );  \\
%TMP\vspace{3mm}
%TMP\\
%TMPFor example, @smalldomain = ( 101, 140, 51, 90 ) 
%TMPtranslates to the following in \Par:\\
%TMP\begin{small}\begin{verbatim}
%TMP  integer, public, parameter ::  &
%TMP  , ISMBEG      =   101    & ! Starting x-coodinate
%TMP  , JSMBEG      =   51     & ! Starting y-coodinate
%TMP  , GIMAX       =   40     & ! Number of global points in longitude
%TMP  , GJMAX       =   40     & ! Number of global points in latitude
%TMP\end{verbatim}
%TMP\end{small}
%TMP
%TMP
%TMP
%TMPThe area run by the model does not need to be the EMEP area, just
%TMPany sub-domain of the meteorology domain. For standard EMEP runs
%TMPthe two will be set equal, but for testing we could choose just to
%TMPhave a tiny run-domain. For testing with 4 or 8 processors then
%TMP\@smalldomain = ( 51, 160, 31, 120 ) works quite well.
%TMP
%TMP\subsection*{Shadow area}
%TMP
%TMPIMPORTANT - the grid cells on the boundaries of the run-area are
%TMPexcluded from the chemical calculations, since the advection routines
%TMPrequire one cell more than the chemistry. Thus, in \Par\ you will find:
%TMP
%TMP\begin{small}\begin{verbatim}
%TMPFrom par_ml.f90
%TMP     limax       & ! Actual number of local points in longitude
%TMP   , ljmax         ! Actual number of local points in latitude
%TMP
%TMP     li0  & ! First local index in longitude when outer boundary is excluded
%TMP   , li1  & ! Last local index in longitude when outer boundary is excluded
%TMP   , lj0  & ! First local index in latitude when outer boundary is excluded
%TMP   , lj1    ! Last local index in latitude when outer boundary is excluded
%TMP\end{verbatim}
%TMP\end{small}
%TMP
%TMP
%TMPSo, if you loop over something that does not need
%TMPinfo from neighboring boxes then it is ok to use li0:li1, lj0:lj1,
%TMPbut if you need info from the "shadow area" (ie advection)  you must use
%TMP1:limax,1:ljmax.
%TMP\bigskip
%TMP\bigskip
%TMP
%TMP
%TMPThe domains used for the outputs are discussed more in section~\ref{OUTPUTS}.
%TMP
