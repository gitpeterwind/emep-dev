\chapter{Input files}
\label{ch:InputFiles}

This chapter provides an overview on the necessary input files to run 
the Unified EMEP model. A complete set of input files is provided in
the EMEP Open Source web page to allow model runs for the year 2008. 

The input files are zipped in 13 different files. 
There are 12 zipped files for the meteorology, one for each month, called meteo2008MM.tar.bz2 (where MM 
is the month). 
The last file for December also include a met file for January 1$^{st}$ 2009. 
The last zipped input file (called other\_input\_files.tar.bz2) contains all other input 
files needed for running the EMEP model, except AircraftEmis\_FL.nc.
After unzipping all of the tar files, there will be two directories under EMEP\_Unified\_model.OpenSource2011. 
All the meteorology are placed under /met, and the rest of the input files under /input. 
All the files, both meteorological and 
the other input files, are described in this chapter.

{\bf IMPORTANT:} The input data available in the EMEP Open Source Web
site should be appropriately acknowledged when used for model runs.
If nothing else is specified according to references further in this
chapter, please acknowledge EMEP/MSC-W in
any use of these data.



\begin{table}
\caption[List of input data files]{List of input data files.
Note: YYYY: year, MM: month, DD: day, SS: seasons, POLL: pollutant
type (NH$_3$, CO, NO$_x$, SO$_x$, NMVOC,
PM$_{2.5}$ and PM$_{co}$). 
\label{Tab:inputdata}}
\begin{center}
\begin{small}
% \hspace{-1cm}
\begin{tabular}{lll}
\hline
{\bf Data} &  {\bf Name} & {\bf Format}\\
{\bf Meteorology data} & met/&  \\
Meteorology  &  meteoYYYYMMDD.nc \quad (366+1 files) & netCDF\\
& & \\
{\bf Other Input files} & input/ &\\
Global Ozone & GLOBAL\_O3.nc & netCDF\\
BVOC emissions & EMEP\_EuroBVOC.nc & netCDF\\
Aircraft emissions & AircraftEmis\_FL.nc & netCDF$\dagger$ \\
Surface Pressure & SurfacePressure.nc & netCDF$\dagger$ \\
& & \\
Landuse & Inputs.Landuse & ASCII$^*$\\
Land/Sea mask & landsea\_mask.dat & ASCII\\
Natural SO$_2$ & natso2MM.dat  \quad (12 files) & ASCII\\
Volcanoes & VolcanoesLL.dat & ASCII$^*$\\
Emissions & emislist.POLL  \quad (7 files) & ASCII\\
Time factors for monthly emissions& MonthlyFac.POLL  \quad (7 files) & ASCII\\
Time factors for daily emissions &  DailyFac.POLL  \quad (7 files) & ASCII\\
Landuse definitions & Inputs\_LandDefs.csv & ASCII$^*$\\
Stomatal conductance & Inputs\_DO3SE.csv & ASCII$^*$\\
Lightning emissions & lightningMM.dat  \quad (12 files) & ASCII$^*$\\
Emissions speciation & emissplit.defaults.POLL & ASCII$^*$\\
                     & emissplit.specials.POLL & ASCII$^{*,\dagger}$\\
Photo-dissociation rates & jclearSS.dat \quad (4 files) & ASCII\\
 & jcl1kmSS.dat \quad (4 files) + jcl1.jun & ASCII\\
 & jcl3kmSS.dat \quad (4 files) + jcl3.jun & ASCII\\
Dust files & sand\_frac.dat clay\_frac.dat & ASCII$\dagger$\\
 & DUST\_f.MM DUST\_c.MM & ASCII$\dagger$\\
Sites locations for surface output & sites.dat & ASCII$^*$\\
Sondes locations for vertical output & sondes.dat & ASCII$^*$\\
Emission factors for scenario runs & femis.dat & ASCII\\
\hline
\end{tabular}\\
Notes: $\dagger$ - optional (in most cases); 
$^*$ means ASCII files with header.
\end{small}
\end{center}

\end{table}

\newpage
\section{NetCDF files}



\subsection{Meteorology}

The daily meteorological input data (``meteoYYYYMMDD.nc'', where YYYY is year, MM is month 
and DD is day) used for the Unified EMEP Model are based on
forecast experiment runs with the Integrated Forecast System (IFS), a global
operational forecasting model from the European Centre for Medium-Range
Weather Forecasts (ECMWF).

The IFS forecasts has been run by MSC-W as
independent experiments on the HPCs at ECMWF with special requests on
some output parameters. The meteorological fields are retrieved on a
0.2$^{\circ}\times$0.2$^{\circ}$ rotated spherical grid and interpolated to
50$\times$50 km$^2$ polar-stereographic grid projection. Vertically, the fields
on 60 eta levels from the IFS model are interpolated onto the 20 EMEP sigma
levels. The version of the IFS model used for preparing these fields,
Cycle 36r1, is documented in \url{http://www.ecmwf.int/research/ifsdocs/index.html}.
Meteorological fields currently used for Unified EMEP Model runs are given in
Table~\ref{Tab:metinput}. Some verification and description of these
meteorological fields are given in Chapter 7 of the EMEP Status Report
1/2010.

{\bf Acknowledgement:} ECMWF, met.no

\begin{table}[h!]
\caption{Input meteorological data used in the Unified EMEP Model
   \label{Tab:metinput}}
\begin{center}
\begin{tabular}{p{6cm}lll}
\hline
Parameter      & Unit & Description          \\
\hline
\multicolumn{3}{l}{3D fields - for 20 $\sigma$ levels} \\
$u,v$  &  m/s     & Wind velocity components   \\
$q$    &  kg/kg   & Specific humidity           \\
$\dot{\sigma}$ & s$^{-1}$ & Vertical wind in $\sigma$ coordinates \\
$\theta$       & K  & Potential temperature \\
$CL$             & \% & 3D Cloud cover            \\
$PR$             & mm & Precipitation         \\
$CW$             & kg/kg & Cloud water          \\
\hline
\multicolumn{3}{l}{2D fields - for Surface} \\
$PS$             & hPa & Surface pressure                     \\
$T2$          & K  & Temperature at 2m height               \\
$Rh2$             & \% & Relative humidity at 2m height$\ast$ \\
SH              & W m$^{-2}$ & Surface flux of sensible heat \\
LH             & W m$^{-2}$ & Surface flux of latent heat \\
$\tau$         & N m$^{-2}$ & Surface stress               \\
SST            & K & Sea surface temperature \\
SWC            & m$^3$/m$^{3}$ & Soil water content$^\ast$        \\
DSWC           & m$^3$/m$^{3}$ & Deep soil water content$^\ast$   \\
sdepth         & m & Snow depth \\
ice            & \% & Fraction of ice \\  
\hline
\end{tabular}\\
Notes: $\ast$ not yet used
\end{center}
\end{table}



\subsection{Global Ozone}

Initial concentration of ozone are required in order to
initialize the model runs. Boundary conditions along the sides of the model
domain and at the top of the domain are then required as the model is
running.

The GLOBAL\_O3.nc file contains monthly averaged fields in netCDF format. 
The initial and background
concentrations are based on the Logan (1998) climatology. The Logan
climatology is scaled by Unimod according to the Mace Head measurements as
described in Simpson {\sl et al.} (2003). For a number of other species, 
background/initial conditions are set within the model using functions 
based on observations (Simpson {\sl et al.}, 2003 and Fagerli {\sl et al.}, 2004).



\subsection{BVOC emissions}

Biogenic emissions of isoprene and monoterpene are calculated in the
model as a function of temperature and solar radiation, using the landuse
datasets. The light and temperature depencies are similar to those
used in the original Open-Source model, see 
Chapter 4.2 of the EMEP Status Report 1/2003 Part I (Simpson
{\sl et al.}, 2003).

Biogenic VOC emission potentials (i.e. rates at 30$^\circ$C and full sunlight)
are included for four different forest types in the netCDF file 
EMEP\_EuroBVOC.nc. These emission potentials have unit $\mu g/m^{2} /h$, and
refer to emissions per area of the appropriate forest category. In 
addition, default emission potentials are given for other
land-cover categories in the file Inputs\_LandDefs.csv. 
The underlying emission potentials, land-cover data bases, and model
coding have however changed substantially for rv3\_7. The new approach
is documented in Simpson {\sl et al.}, 2011.






\subsection{Aircraft emissions}
In the Unified EMEP model aircraft emissions are 'OFF' by default. 
They can be switched 'ON' by setting USE\_AIRCRAFT\_EMIS = .true. in { \bf ModelConstants\_ml.f90.} 
The model will then use data provided by the EU-Framework Programme 6 Integrated 
Project QUANTIFY (\url{http://www.pa.op.dlr.de/quantify}). However, before using 
these data a protocol has to be signed, which is why the data file can not be provided 
directly on the EMEP Open Source website. If you want to use aircraft emissions go to 
\url{http://www.pa.op.dlr.de/quantify}, click on 'QUANTIFY emission inventories and scenarios', 
and then click on 'Register'. That page will provide information about the registration 
process and the protocol that has to be signed. Once you are registered, click 'Login' and 
provide user name and password. On the new page, search for 'Emissions for EMEP', which 
links directly to the Readme file and the emission data file in netCDF format. Download the 
emission data file and place it in the input folder.

\subsection{Surface Pressure}

If USE\_AIRCRAFT\_EMIS = .true. in { \bf ModelConstants\_ml.f90,} then in addition to the Aircraft 
Emission file, there will be need for a SurfacePressure.nc file, which is already in the /input folder. 
The netCDF file consists of surface pressure fields for each of the months in 2008 called surface\_pressure, 
and one field for the whole year called surface\_pressure\_year. All fields are given in Pa. 

\section{ASCII files}

\subsection{Landuse}

Landuse data are required for modeling boundary layer processes
(i.e. dry deposition, turbulent diffusion).
The Unified EMEP model can accept landuse data from any
data set covering the whole of the domain, providing reasonable 
resolution of the vegetation categories. Gridded data sets providing
these landuse categories across the EMEP domain have been created
based on the data from the Stockholm Environment Institute at York 
(SEI-Y) and from the Coordinating Center for Effects (CCE). 
16 basic landuse classes have been identified for use in the
deposition module in the model, and three additional``fake'' landuse
classes are used for providing results for integrated assessment
modeling and effects work.

There is a {\bf header} in the file that contains short abbreviations 
for the different land cover
types, e.g. CF for temperate/boreal coniferous forest. The landuse
types are summarized in Table 5.1 in Chapter 5 of the EMEP Status
Report 1/2003 Part I (Simpson {\sl et al.}, 2003).

The different landuse types are given as a percentage of area for each 
EMEP grid cell in the {\bf ASCII file} ``Inputs.Landuse''. 



\subsection{Land/Sea mask}
This file, based on the meteorological model's roughness length, is used
to assign a land/sea mask within the Unified EMEP model, since there is
a need to modify the stability information for coastal grid cells
(which contain land, but is not resolved by the NWP model supplying 
meteorological data to the Unified EMEP model). 

The gridded {\bf ASCII} file ``landsea\_mask.dat'' contains 3 columns. 
The first two columns represent the `i' and `j' indices of the EECCA
grid and the third column gives the class of the grid cell, where the 
EMEP model use 0 for ocean, and $\geq$ 1 as land.



\subsection{Natural $SO_2$}
Natural $SO_2$ emissions (dimethylsulfide (DMS) from sea) are provided 
as monthly gridded files.  
The values are given at the surface in $\mu$g/m$^2$ for each grid cell in the domain. 

\subsection{Volcanoes}

Emissions from volcanoes are included for Italy, based upon the
officially submitted data.
To consider the volcanic emissions, we need to feed the location
and height of volcanoes into the model. 

The input file ``VolcanoesLL.dat'' in {\bf ASCII} format contains the input data for the two active volcanoes
 Enta and Stromboli. The two first columns gives their latitudes and longitudes, their height as sigma level
 in column 3 and their emission in ktons per year in column 4. 

\subsection{Gridded Emissions}
The emission input for the Unified EMEP model consists of gridded
annual national emissions based on emission data reported every year
to EMEP/MSC-W (until 2005) 
and to CEIP (from 2006) by each
participating country. 
More details about the emission input with references can be
found in Chapter 4 of the EMEP Status Report 1/2003 Part I 
(Simpson et al., 2003).

The 7 gridded emission input files ``emislist.poll'' are available in 
{\bf ASCII} format for the following compounds: CO, NH$_{3}$,
NO$_{x}$, PM$_{2.5}$, PM$_{co}$, SO$_{x}$ and VOC.

The gridded emission files contain 16 columns where the first column 
represents the country code
(\url{http://www.emep.int/grid/country_numbers.txt}), 
the second and the third columns are the `i' and `j' indices of the
EMEP grid, the fourth and fifth columns are the total emissions from
low and high sources, and the last 11 columns contain emissions from 
10 anthropogenic SNAP sectors 
(\url{http://reports.eea.eu.int/technical_report_2001_3/en}) and 1 
source-sector called``Other sources and sinks'', which include natural and
biogenic emission sources. The data are given with the unit: $Mg$.

{\bf Acknowledgement:} EMEP

\subsection{Time factors for emissions}

Monthly and daily time factors for emission of the 7 compounds 
(CO, NH$_{3}$, NO$_{x}$, PM$_{2.5}$, PM$_{co}$, SO$_{x}$ and VOC). 
There is one file available per compound in {\bf ASCII} format. 

The first two columns in the files represent the country code \\
(\url{http://www.emep.int/grid/country_numbers.txt}), the second column 
represents the sector (\url{http://webdab.emep.int/sectors.html}). In the monthly files, 
the 12 consecutive columns represent the time factors corresponding to 
the months of the year. In the daily files there are 7 consecutive columns representing 
the time factor for each day of the week. 

% \COMMENT{I moved all emission-related factors here}

\subsection{Emission factor for scenario runs}\label{sec:femis}
Scenario run in the case of the Unified EMEP model means a run to test
the impact of one or more pollutants from a particular
country. 

Emission factors are applied to specified countries and
emission sectors and can be set by changing the {\bf ASCII} file
``femis.dat''. 
This file can be changed by the users according to their needs.

The file contains several columns (the number is flexible). The first column represents
the country code (\url{http://www.emep.int/grid/country_numbers.txt}),
the second represents the sector
(\url{http://reports.eea.eu.int/technical_report_2001_3/en}) 
where `0' means all sectors, and then in the remaining
 columns one can specify
which emissions to reduce. Here 1.0 means no reduction of the given
pollutant 
(sox/nox/voc/nh3/co/pm25/pmco) from sectors of specified country. The
number following the first text (``Name'') in line 1 (number 5 in
the downloaded file) gives the number of pollutants treated in the file.


\subsection{Chemical speciation of emissions}

%\NEW{
Many of the emission files give emissions of a group of compounds, e.g.
NOx includes NO+NO$_2$, and VOC can include many compounds. The information
needed to retreive emissions of individual compounds from these the
gridded files is given in  files labelled emissplit.defaults.$poll$ or
emissplit.specials.$poll$, where $poll$ can be nox, voc, etc.

The defaults file give the emission split for each SNAP sector (
one per row, with second index being the SNAP sector), which
is applied to all countries by default. For VOC this split
was derived from the UK inventory of Passant (2002),
as part of the chemical comparison project of Hayman {\sl et al.} (2011).

The specials files are in general optional, and can be used to specify
speciation for particular countries or SNAP sectors. The
1${^st}$ column specifies the country code of interest, the second the SNAP sector. 

If forest fires are used, then the file emissplit.specials.voc is required
(not optional), and the country-code 101 used to specify the VOC speciation
of forest fires in this file.
% }

\subsection{Lightning emissions}
Emissions of NO$_{x}$ from lightning are included in the model
as monthly averages on a T21 (5.65$^{\circ}\times$5.65$^{\circ}$) resolution (K{\o}hler {\sl et al.}, 1995). 
The lightning emissions are defined on a 64$\times$32 grid with 17 vertical
levels, with global coverage, and are provided as 12 {\bf ASCII} files
``lt21-nox.datMM''.



\subsection{Landuse definitions}
For the vegetative landuse categories where stomatal modeling is
undertaken, the start and end of the growing season (SGS, EGS, in days) must be specified. 
The calculation of SGS and EGS with respect to latitude is done 
in the module {\bf LandDefs\_ml.f90}. 
The parameters needed to specify the
development of the leaf area index (LAI) within the growing season
are given in the ASCII file ``Inputs\_LandDefs.csv''. 
For more information, see chapter 5  of the EMEP Status Report 1/2003 Part I (Simpson {\sl et al.}, 2003).


The file, designed to be opened with excel or gnumeric,
contains a header briefly explaining the contents of the 
14 columns. 
The first three columns are representing the landuse name, code (which
are consistent with those in ``Landuse.Input'' file) and
type (grouping of the landuse classes). The fourth column 
% \NEW{
(PFT) gives a plant-functional type code (for future use),
 the fifth gives
the maximum height of vegetation ($m$), the sixth indicates albedo (\%) and
the seventh indicates possible source of NH$_{x}$ (0 off/1 on,
curently not used). Columns 8 to 11
define the growing season (day number), column 12 and 13 lists the
LAI minimum  and maximum ($m^{2}/m^{2}$) and 
columns 14 and 15
defines the length of the LAI increase and decline periods (no. of days).
% \NEW{
Finally, the last four columns give default values of 
foliar biomass and biogenic VOC emission potentials. See Simpson
et al., 2011 for details.
% }


\subsection{Stomatal conductance}
Parameters for the stomatal conductance model, deposition of O$_{3}$ and
stomatal exchange (DO3SE) must be specified. That are based upon the ideas in
Emberson {\sl et al.}, 2000, and are discussed in Simpson and Emberson,
2006 and Tuovinen et al. 2004.

The  ASCII file ``Inputs\_DO3SE.csv'' provides land-phenology data
of each landuse type for stomatal conductance calculations. The 
data are summarised in Table 5.1 in Chapter 5 of the EMEP 
Status Report 1/2003 Part I (Simpson {\sl et al.}, 2003).

The file contains a {\bf header} with the contents of the file,
with different factors needed for each of the landuse classes used
in the Unified EMEP model. The first two columns represent the
landuse code (which are consistent with those in ``Landuse.Input'' file)
and name. The next 22 values are different phenology factors.


\subsection{Photo-dissociation rates}
The photo-dissociation rates (J-values) are provided as lookup
tables. The method is previously described in Jonson {\sl et
al.}, (2001). J-values are provided as clear sky, light cloud and dense
cloud conditions, and the model interpolates between these according
to cloudiness from the meteorological input data. In the lookup tables
data are listed for every 10 degree latitude at an interval of 1
degree zenith angle at every model height.

For the two types of cloud conditions there are one {\bf ASCII} file 
averaged for each season (SS); 01, 02, 03 and 04. 
For light cloud the four seasonal files are called ``jcl1kmSS.dat'', for dense cloud conditions
the four seasonal files are called ``jcl3kmSS.dat'', and then for clear sky four files 
called ``jclearSS.dat''. In addittion there are wo files for june called jcl1.jun 
and jcl3.jun.

Each file contains 18 columns. The first column is latitude of zenith 
angle and then the
next 17 are the values for the model levels with the unit: $1/s$. 
For more details about these rates, please read Chapter 7.2 of the EMEP
Status Report 1/2003 Part I (Simpson {\sl et al.}, 2003).

\subsection{Dust files}

The gridded input files ``sand\_frac.dat'' and ``clay\_frac.dat'' gives the fraction (in \%)  
of sand an clay in the soil for each grid cell over land. The files are provided by 
Maxilian Posch, Netherlands Environmental Assessment Agency, CCE, personal communication.

The files are used by the module {\bf DustProd\_ml.f90 }, which calculates windblown dust 
emissions from soil erosion. Note that the parametrization is still in the development and 
testing phase, and is by default 'turned off'. To include it in the model calculations, set 
USE\_DUST = .true. in {\bf ModelConstants\_ml.f90}.
The user is recommended to read carefully documentation and
comments in the module {\bf DustProd\_ml.f90}.

There is also a possibility to include boundary and initial conditions for dust from Sahara. 
The gridded input files ``DUST\_f.MM'' and ``DUST\_c.MM'' gives monthly dust mixing 
ratios (MM - month, e.g. 01, 02, 03,...) for fine and coarse dust from Sahara. The files are based on calculations 
from a global CTM at the University of Oslo for 2000. 
To include Saharan dust, set USE\_SAHARA = .true. in {\bf ModelConstants\_ml.f90}.



\subsection{Site and Sonde locations for output}\label{sec:sitessondes_input}
The model provides a possibility for extra output data of surface concentration 
for a set of specified measurement site locations and concentrations for the vertical 
column above a set of specified locations. These site and sonde locations are listed 
in the {\bf ASCII} files ``sites.dat`` and ``sondes.dat`` files. These files can be 
changed by the user, this is described in section \ref{sec:sitesonde}.

