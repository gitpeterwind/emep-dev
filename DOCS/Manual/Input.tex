\chapter{Input files}
\label{ch:InputFiles}

This chapter provides an overview on the necessary input files to run 
the Unified EMEP model. A complete set of input files is provided in
the EMEP Open Source web page to allow model runs for the year 2005.

{\bf IMPORTANT:} The input data available in the EMEP Open Source Web
site should be appropriately acknowledged when used for model runs.
If nothing else is specified according to references further in this
chapter, please acknowledge EMEP/MSC-W in
any use of these data.

%%litt om filer og grid

%%asci vs netcdf

%tabell


\begin{table}
\caption[List of input data files]{List of input data files.
Note: YYYY: year, MM: month, DD: day, SEASON: seasons, POLL: pollutant
type (NH$_3$, CO, NO$_x$, SO$_x$, NMVOC,
PM$_{2.5}$ and PM$_{co}$). ASCII$^*$ means ASCII files with header.
\label{Tab:inputdata}}
\begin{small}
\hspace{-1cm}
\begin{tabular}{lll}
 && \\
\hline
{\bf Data} &  {\bf Name} & {\bf Format}\\
\hline\hline
% && \\
% {\bf Meteorology data directory} & EMEP\_metdata/YYYY/&  \\
\hline
Meteorology  &  meteoYYYYMMDD.nc \quad (366+1 files) & netCDF\\
& & \\
Boundary and initial conditions & Boundary\_and\_Initial\_Conditions.nc & netCDF\\
Landuse & Inputs.Landuse & ASCII$^*$\\
%BVOC emissions & Inputs.BVOC & ASCII$^*$\\
Land/Sea mask & landsea\_mask.dat & ASCII\\
Sites locations for surface output & sites.dat & ASCII$^*$\\
Sondes locations for vertical output & sondes.dat & ASCII$^*$\\
%Snow cover & snowcMM.dat.170  \quad (12 files) & ASCII\\
Natural SO$_2$ & natso2MM.dat  \quad (12 files) & ASCII\\
Volcanoes & Volcanoes.dat & ASCII$^*$\\
&& \\
{\bf Emission data directory } & EMEP\_GriddedData/Emissions/&  \\
\hline
Emissions & GridPOLL  \quad (7 files) & ASCII\\
&& \\
{\bf Grid-independent data directory} & Common/ & \\
\hline
Time factors for monthly emissions& MonthlyFac.POLL  \quad (7 files) & ASCII\\
Time factors for daily emissions &  DailyFac.POLL  \quad (7 files)
& ASCII\\
Landuse definitions & Inputs\_LandDefs.csv & ASCII$^*$\\
Stomatal conductance & Inputs\_DO3SE.csv & ASCII$^*$\\
Lightning emissions & lt21-nox.datMM  \quad (12 files) & ASCII$^*$\\
%Military aircraft emissions & amilt42-nox.dat & ASCII\\
%Commercial aircraft emissions & aSEASONt42-nox.dat \quad (4 files)& ASCII\\
Aircraft emissions & AircraftEmis\_FL.nc & netCDF \\
VOC speciation & vocsplit.defaults.BASE\_MAR2004 & ASCII\\
 & vocsplit.special.BASE\_MAR2004  & ASCII\\
NOx speciation & noxsplit.default.2000 & ASCII\\
 &noxsplit.special.BASE\_MAR2004& ASCII\\  
Emission factors for scenario runs & femis.dat & ASCII\\
Photo-dissociation rates & jclear.SEASON \quad (4 files) & ASCII\\
 & jcl1.SEASON \quad (4 files) & ASCII\\
 & jcl3.SEASON \quad (4 files) & ASCII\\
\hline
\end{tabular}
\end{small}


\end{table}

\subsection{NetCDF files}


The 3-hourly meteorological input data and the boundary and initial 
conditions input files are provided in netCDF format. 
We presently follow as much as possible the netCDF CF1.0 conventions 
(\url{http://www.unidata.ucar.edu/software/netcdf/}) for both input
and output data (except the Boundary\_and\_Initial\_Conditions file which
is still in GDV convention).

\subsubsection{Meteorology}

The meteorology input data is from the Integrated Forecast System (IFS), an global operational 
forecasting model from the European Centre for Medium-Range Weather Forecasts (ECMWF). 

The daily ``meteoYYYYMMDD.nc'' files with 3-hourly values are in netCDF 
(CF1.0 convention) format.

The parameters extracted from the ECMWF Meteorological output
currently used for Unified EMEP Model runs are given in 
Table~\ref{Tab:metinput}. A further description of these
meteorological fields is given in Chapter 3 of the EMEP Status Report
1/2003 Part I (Simpson {\sl et al.}, 2003).

{\bf Acknowledgement:} met.no

Please acknowledge The Norwegian Meteorological Institute in all
publications using these data.
 
\begin{table}[hb]
\caption{Input meteorological data used in the Unified EMEP Model
   \label{Tab:metinput}}
\begin{tabular}{p{6cm}lll}
\hline
Parameter      & Unit & Description          \\
\hline
\multicolumn{3}{l}{3D fields - for 20 $\sigma$ levels} \\
$u,v$  &  m/s     & Wind velocity components   \\
$q$    &  kg/kg   & Specific humidity           \\
$\dot{\sigma}$ & s$^{-1}$ & Vertical wind in $\sigma$ coordinates \\
$\theta$       & K  & Potential temperature \\
$CL$             & \% & 3D Cloud cover            \\
$PR$             & mm & Precipitation         \\
\hline
\multicolumn{3}{l}{2D fields - for Surface} \\
$P_s$             & hPa & Surface pressure                     \\
$T_2$          & K  & Temperature at 2m height               \\
$Rh_2$             & \% & Relative humidity at 2m height \\
SH              & W m$^{-2}$ & Surface flux of sensible heat \\
LH             & W m$^{-2}$ & Surface flux of latent heat \\
$\tau$         & M m$^{-2}$ & Surface stress               \\
SST            & K & Sea surface temperature \\
sdepth         & m & Snow depth \\
ice            & \% & Fraction of ice \\  
\hline
\end{tabular}
\end{table}

\subsubsection{Boundary and Initial Conditions}

Monthly files

D3\_O3\_Logan, D3\_CH3COO2,D3\_H2O2,D3\_OH

\subsection{Landuse}

Landuse data are required for modelling booundary layer processes
(i.e. dry depositon, turbulent diffusion).
The Unified EMEP model can accept landuse data from any
data set covering the whole of the domain and providing reasonable 
resolution of vegetation categories. Gridded data sets providing
these landuse categories across the EMEP domain have been created
based on the data from the Stockholm Environment Institute at York 
(SEI-Y) and from the Coordinating Centre for Effects (CCE). 
16 basic landuse classes have been identified for use in the
deposition module in the model, and three additional``fake'' landuse
classes are used for providing results for integrated assessment
modelling and effects work.

There is a {\bf header} in the file that contains short abbreviations 
for the different landcover
types, e.g. CF for temperate/boreal coniferous forest. The landuse
types are summarised in Table 5.1 in Chapter 5 of the EMEP Status
Report 1/2003 Part I (Simpson {\sl et al.}, 2003).

The different landuse types are given as a percentage of area for each 
EMEP grid cell in the {\bf ASCII file} ``Inputs.Landuse''. 



\subsection{Land/Sea mask}
This file, based on HIRLAM roughness length is used
to assign land/sea mask within the Unified EMEP model, since there is
a need to modify the stability information for coastal grid cells
(which contain land, but is not resolved by the NWP model supplying 
meteorological data to the Unified EMEP model). 

The gridded {\bf ASCII} file ``landsea\_mask.dat'' contains 3 columns. 
The first two columns represent the `i' and `j' indices of the EECCA
grid and the third column gives the class of the grid cell, where the 
EMEP model use 0 for ocean, and $\geq$ 1 as land.% length with unit: $m$.

\subsection{Site and Sonde locations for output}\label{sec:sitessondes_input}
The model provides a possibility for extra output data of surface concentration 
for a set of specified measurement site locations and concentrations for the vertical 
column above a set of specified locations. These site and sonde locations are listed 
in the {\bf ASCII} files ``sites.dat`` and ``sondes.dat`` files. These files can be 
changed by the user, this is described in section \ref{sec:sitesonde}.


\subsection{Natural $SO_2$}
Natural $SO_2 emissions$ (dimenthylsulfide (DMS) from sea) are provided 
an monthly gridded files. They are stored in columns of `i' and `j' and the 12 monthly `values'. 
The values are given at the surface in $\mu g /m³ $ for each grid cell in the domain. 

\subsection{Volcanoes}

Emissions of volcanoes are included for Italy, based upon the
officially submitted data.
To consider the volcanic emissions, we need to feed the location
and height of volcanoes into the model. 

The input file ``Volcanoes.dat'' in {\bf ASCII} format contains the input data for the two active volcanoes
 Enta and Stromboli. The two first colums gives their latitudes and longitudes, their height as sigma level
 in column 3 and their emission in ktons per year in column 4. 

\subsection{Emissions}
The emission input for the Unified EMEP model consists of gridded
annual national emissions based on emission data reported every year
to EMEP/MSC-W (until 2005) 
and to CEIP (from 2006) by each
participating country. 
More details about the emission input with references can be
found in Chapter 4 of the EMEP Status Report 1/2003 Part I 
(Simpson et al., 2003).

The 7 gridded emission input files ``gridPOLL'' are available in 
{\bf ASCII} format for the following compounds: CO, NH$_{3}$,
NO$_{x}$, PM$_{2.5}$, PM$_{co}$, SO$_{x}$ and VOC.

The gridded emission files contain 16 columns where the first column 
represents the country code
(\url{http://www.emep.int/grid/country_numbers.txt}), 
the second and the third columns are the `i' and `j' indices of the
EMEP grid, the fourth and fifth columns are the total emissions from
low and high sources, and the rest 11 columns contain emissions from 
10 anthropogenic SNAP sectors 
(\url{http://reports.eea.eu.int/technical_report_2001_3/en}) and 1 
source-sector called``Other sources and sinks'', which include natural and
biogenic emission sources. The data are given with the unit: $Tonnes$.

{\bf Acknowledgement:} EMEP

\subsection{Time factors for emissions}

Monthly and daily time factors for emission of the 7 compounds 
(CO, NH$_{3}$, NO$_{x}$, PM$_{2.5}$, PM$_{co}$, SO$_{x}$ and VOC). 
There is one file available per compound in {\bf ASCII} format. 

The first two columns in the files represent the country code
(\url{http://www.emep.int/grid/country_numbers.txt}), the second column 
represents the sector (\url{http://webdab.emep.int/sectors.html}. In the monthly files, 
the 12 consecutive columns represent the time factors corresponding to 
the months of the year. In the daily files there are 7 consecutive columns representing 
the time factor for each day of the week. 

\subsection{Landuse definitions}
For those vegetative landuse categories for which stomatal modelling is
undertaken, the start and end of the growing season (SGS, EGS) with
latitude dependent data to capture north-south differences and the
development of the leaf area index (LAI) within this growing season
must be specified. There is a schematic of LAI development in Figure
5.1 in Chapter 5 of the EMEP Status Report 1/2003 Part I (Simpson {\sl et al.}, 2003).

The {\bf ASCII} file ``Inputs\_LandDefs.csv'' provides land-phenology data
of each landuse type summarised in Table 5.1 in Chapter 5 of the EMEP 
Status Report 1/2003 Part I (Simpson {\sl et al.}, 2003) for dry deposition 
calculations.

The file contain a {\bf header} explaing briefly the contents of the 
14 columns. 
The first three columns are representing the landuse name, code (which
are consistent with those in ``Landuse.Input'' file) and
type (grouping of the landuse classes). The fourth column gives
the maximum height of vegetation ($m$), the fifth indicates albedo (\%) and
the sixth indicates possible source of NH$_{x}$ (0/1). 7 to 10
column is about growing season (day number), 11 and 12 indicates the
LAI minimum  and maximum ($m^{2}/m^{2}$) and then 13 and 14 column
represent the length of the LAI increase and decline periods (no. of days).


\subsection{Stomatal conductance}
Parameters for the stomatal conductance model, deposition of O$_{3}$ and
stomatal exchange (DO3SE) that are based upon the ideas in
Emberson {\sl et al.}, 2000, and further discussed in Simpson and Emberson,
2006, must be specified.  

The {\bf ASCII} file ``Inputs\_DO3SE.csv'' provides land-phenology data
of each landuse type summarised in Table 5.1 in Chapter 5 of the EMEP 
Status Report 1/2003 Part I (Simpson {\sl et al.}, 2003) for stomatal
conductance calculations.

The file contains a {\bf header} indicating the contents of the file
with the different factors needed for each of the landuse classes used
in the Unified EMEP model. The first two columns are representing the
landuse code (which are consistent with those in ``Landuse.Input'' file)
and name. The next 22 values are different phenology factors.

\subsection{Lightning emissions}
Emissions of NO$_{x}$ from lightning are included in the model
as monthly averages on a T21 (5.65$^{\circ}$$\times$5.65$^{\circ}$) resolution (K{\o}hler {\sl et al.}, 1995). 
The lightning emissions are defined on a 64$\times$32 grid with 17 vertical
levels, with global coverage and are provided as 12 {\bf ASCII} files
``lt21-nox.datMM''.


\subsection{Aircraft emissions}
In the Unified EMEP model aircraft emissions are 'OFF' by default. 
They can be switched 'ON' by setting USE\_AIRCRAFT\_EMIS = TRUE in ModelConstants\_ml.f90. 
The model will then use data provided by the EU-Framework Programme 6 Integrated 
Project QUANTIFY (\url{http://www.pa.op.dlr.de/quantify}). However, before using 
these data a protocol has to be signed, which is why the data file can not be provided 
directly on the EMEP Open Source website. If you want to use aircraft emissions go to 
\url{http://www.pa.op.dlr.de/quantify}, click on 'QUANTIFY emission inventories and scenarios', 
and then click on 'Register'. That page will provide information about the registration 
process and the protocol that has to be signed. Once you are registered, click 'Login' and 
provide username and password. On the new page, search for 'Emissions for EMEP', which 
links directly to the Readme file and the emission data file on netCDF format. Download the 
emission data file and place it in the input folder.

\subsection{VOC speciation}
Speciation of VOC emissions are specified for each source-sector,
derived from the detailed United Kingdom speciation given in PORG
(1993). The Unified EMEP model uses a `lumped-molecule' approach to
VOC emissions and modelling, in which for example model species NC4H10
represents all C3+ alkanes, and o-xylene represents all aromatic
species. Therefore, each of the species from the detailed UK inventory
has been assigned to one of the Unified EMEP model's species according
to its reactivity and chemical composition, as given in
Andersson-Sk\"{o}ld and Simpson (1997). Although the exact VOC speciation
used can be varied (``vocsplit.special.BASE\_MAR2004'') to suit
particular emission scenarios (e.g. Reis {\sl et al.}, 2000), a default split
(``vocsplit.defaults.BASE\_MAR2004'') is typically used, as given in
Table 4.3 in Chapter 4 of the EMEP Status Report 1/2003 Part I (Simpson {\sl et al.}, 2003).

There are two available {\bf ASCII} files which one can choose to
run the model with, depending on whether a particular scenario should be
suited or not. However, the files
``vocsplit.defaults.BASE\_MAR2004'' and ``vocsplit.special.BASE\_MAR2004''
have currently the same content.

\subsection{NOx speciation}
Speciation of NO$_x$ emissions are specified for each source-sector and
for each country if the file ``noxsplit.special.BASE\_MAR2004'' is
chosen for the model run. The amount of NO$_{2}$ compared to
other NO is determined by these split-files.

There are two downloadable {\bf ASCII} files which one can choose to
run the model with, depending on whether a particular scenario should be
suited or not.
The file ``noxsplit.default.2000'' is only sector dependent and the file
``noxsplit.special.BASE\_MAR2004'' is both sector and country dependent.

\subsection{Emission factor for scenario runs}
Scenario run in the case of the Unified EMEP model means a run to test
the impact of one or more pollutants from a particular
country. 

Emission factors are applied to specified countries and
emission sectors and can be set by changing the {\bf ASCII} file
``femis.dat''. 
This file can be changed by the users according to their needs.

The file contains 10 columns. The first column represents
the country code (\url{http://www.emep.int/grid/country_numbers.txt}),
the second represents the sector
(\url{http://reports.eea.eu.int/technical_report_2001_3/en}) 
where `0' means all sectors, and then in the next 7 columns one can specify
which emissions to reduce. Here 1.0 means 100\% emissions of the given
pollutant 
(sox/nox/voc/nh3/co/pm25/pmco) from sectors of specified country. The
number following the first text (``Name'') in line 1 (number 5 in
the downloaded file) gives the number of pollutants treated in the file.

\subsection{Photo-dissociation rates}
The photo-dissociation rates (J-values) are provided as lookup
tables. The method is previously described in Jonson {\sl et
al.}, (2001). J-values are provided as clear sky, light cloud and dense
cloud conditions, and the model interpolates between these according
to cloudiness from the meteorological input data. In the lookup tables
data are listed for every 10 degree latitude at an interval of 1
degree zenith angle at every model height.

For the two types of cloud conditions there are one {\bf ASCII} file 
averaged for each season; January, April, July and October (jan represents
winter, apr represents spring, jul represents summer, oct represents
autumn). For light
cloud the four seasonal files are called ``jcl1.SEASON'', for dense cloud conditions
the four seasonal files are called ``jcl3.SEASON'', and then for clear sky the one file 
is called ``jclear.dat''.

Each file contains 18 columns. The first column is latitude of zenith 
angle and then the
next 17 are the values for the model levels with the unit: $1/s$. 
For more details about these rates, please read Chapter 7.2 of the EMEP
Status Report 1/2003 Part I (Simpson {\sl et al.}, 2003).

