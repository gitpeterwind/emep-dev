\chapter{Input files}
\label{ch:InputFiles}

This chapter provides an overview on the necessary input files to run the 
EMEP/MSC-W model. A complete set of input files is provided in the EMEP/MSC-W 
Open Source web page to allow model runs for the meteorological year 2012. 

The input files are zipped in 13 different files. There are 12 zipped files 
for the meteorology, one for each month, called meteo2012MM.tar.bz2 (where MM 
is the month). 
The last file for December also include a met file for January 1$^{st}$ 2013. 
The last zipped input file (called other\_input\_files.tar.bz2) contains all 
other input files needed for running the EMEP/MSC-W model, except the aircraft emissions,
AircraftEmis\_FL.nc, and the forest fire emissions, FINN\_ForestFireEmis\_2012.nc. See sections 2.1.8 and 2.1.10 for details about these emissions data.

After unzipping all the meteo-tar files, the meteorology are placed under a
catalogue called {\bf EMEP\_Unified\_model.OpenSource2014/}. The met/
catalogue should be moved over to current version of model, for rv4\_5:
{\bf EMEP\_MSC-W\_model.rv4\_5.OpenSource}. 
So now there will be two directories under 
{\bf EMEP\_MSC-W\_model.rv4\_5.OpenSource}. 
All meteorology are placed under /met, and the rest of the input files under 
/input. 
All files, both meteorological and the other input files, are described in 
this chapter.

{\bf IMPORTANT:} The input data available in the EMEP/MSC-W Open Source Web
site should be appropriately acknowledged when used for model runs.
If nothing else is specified according to references further in this
chapter, please acknowledge EMEP/MSC-W in any use of these data.

\begin{table}
\caption[List of input data files]{List of input data files.
Note: YYYY: year, MM: month, DD: day, SS: seasons, POLL: pollutant
type (NH$_3$, CO, NO$_x$, SO$_x$, NMVOC,
PM$_{2.5}$ and PM$_{co}$). 
\label{Tab:inputdata}}
\begin{center}
\begin{small}
% \hspace{-1cm}
\begin{tabular}{lll}
\hline
{\bf Data} &  {\bf Name} & {\bf Format}\\
{\bf Meteorology data} & met/&  \\
Meteorology  &  meteoYYYYMMDD.nc \quad (365+1 files) & netCDF\\
& & \\
{\bf Other Input files} & input/ &\\
Global Ozone & GLOBAL\_O3.nc & netCDF\\
New Global Ozone & Logan\_P.nc & netCDF $^*$$^*$\\
Vertical level distribution & Vertical\_levels.txt & ASCII\\
BVOC emissions & EMEP\_EuroBVOC.nc & netCDF\\
Landuse & LanduseGLC.nc and Landuse\_PS\_5km\_LC.nc  & netCDF\\
Degree-day factor & DegreeDayFactors.nc &  netCDF\\
N depositions & annualNdep.nc  & netCDF\\
Road dust &  RoadMap.nc and AVG\_SMI\_2005\_2010.nc& netCDF$\dagger$ \\
Aircraft emissions & AircraftEmis\_FL.nc & netCDF$\dagger$ \\
Surface Pressure & SurfacePressure.nc & netCDF$\dagger$ \\
Forest Fire & FINN\_ForestFireEmis\_YYYY.nc & netCDF$\dagger$ \\
& & \\
% Landuse & Inputs.Landuse & ASCII$^*$\\
% Land/Sea mask & landsea\_mask.dat & ASCII\\
Natural SO$_2$ & natso2MM.dat  \quad (12 files) & ASCII\\
Volcanoes & VolcanoesLL.dat & ASCII$^*$\\
	  & eruptions.csv volcanoes.csv& ASCII \\ 
Emissions & gridPOLL  \quad (7 files) & ASCII\\
Time factors for monthly emissions& MonthlyFac.POLL  \quad (7 files) & ASCII\\
Time factors for daily emissions &  DailyFac.POLL  \quad (7 files) & ASCII\\
Time factors for hourly emissions & HOURLY-FACS  & ASCII$^*$\\
Emission heights & EmisHeights.txt & ASCII$^*$\\
Landuse definitions & Inputs\_LandDefs.csv & ASCII$^*$\\
Stomatal conductance & Inputs\_DO3SE.csv & ASCII$^*$\\
Lightning emissions & lightningMM.dat  \quad (12 files) & ASCII$^*$\\
Emissions speciation & emissplit.defaults.POLL & ASCII$^*$\\
                     & emissplit.specials.POLL & ASCII$^{*,\dagger}$\\
Photo-dissociation rates & jclearSS.dat \quad (4 files) & ASCII\\
 & jcl1kmSS.dat \quad (4 files) + jcl1.jun & ASCII\\
 & jcl3kmSS.dat \quad (4 files) + jcl3.jun & ASCII\\
Dust files  &Soil\_Tegen.nc  & netCDF$\dagger$\\
 & SoilTypes\_IFS.nc & netCDF$\dagger$\\
Sites locations for surface output & sites.dat & ASCII$^*$\\
Sondes locations for vertical output & sondes.dat & ASCII$^*$\\
Emission factors for scenario runs & femis.dat & ASCII\\
\hline
\end{tabular}\\
Notes: $\dagger$ - optional (in most cases); 
$^*$ means ASCII files with header.
$^*$$^*$ New O3 boundary condition data in 30 levels.  Can be used with 'NewLogan=.true.' in 'BoundaryConditions\_ml.f90'.
\end{small}
\end{center}

\end{table}

\newpage


\section{NetCDF files}



\subsection{Meteorology}
The daily meteorological input data (``meteoYYYYMMDD.nc'', where YYYY is year, MM is month 
and DD is day) used for the EMEP/MSC-W Model are based on
forecast experiment runs with the Integrated Forecast System (IFS), a global
operational forecasting model from the European Centre for Medium-Range
Weather Forecasts (ECMWF).

The IFS forecasts has been run by MSC-W as
independent experiments on the HPCs at ECMWF with special requests on
some output parameters. The meteorological fields are retrieved on a
0.1$^{\circ}\times$0.1$^{\circ}$ longitude latitude coordinates and interpolated to
50$\times$50 km$^2$ polar-stereographic grid projection. Vertically, the fields
on 60 eta levels from the IFS model are interpolated onto the 37 EMEP sigma
levels.  The meteorology is prepared into 37 sigma levels since the model is 
under test for a finer vertical resolution.  But the Opensource code is released
with 20 sigma levels and to make the model read the meteorology properly, 
a description of the 20 vertical sigma levels is needed.  This is provided in 
an ascii file called 'Vertical\_levels.txt' together with the other\_input data. 
The version of the IFS model used for preparing these fields,
Cycle 38r2, is documented in \url{http://www.ecmwf.int/research/ifsdocs/index.html}. Previous years are based on Cycle 36r1 with a resolution of 0.2$^{\circ}\times$0.2$^{\circ}$ on a spherical grid. 
Meteorological fields currently used for EMEP/MSC-W Model runs are given in
Table~\ref{Tab:metinput}. Some verification and description of these
meteorological fields are given in Chapter 2 of the EMEP Status Report
1/2014.

{\bf Acknowledgement:} ECMWF, met.no

\newpage
\begin{table}[h!]
\caption{Input meteorological data used in the EMEP/MSC-W Model
   \label{Tab:metinput}}
\begin{center}
\begin{tabular}{p{6cm}lll}
\hline
Parameter      & Unit & Description          \\
\hline
\multicolumn{3}{l}{3D fields - for 37 $\sigma$ levels} \\
$u,v$  &  m/s     & Horizontal wind velocity components   \\
$q$    &  $\nicefrac{kg}{kg}$   & Specific humidity           \\
%$\dot{\sigma}$ & s$^{-1}$ & Vertical wind in $\sigma$ coordinates \\
$\theta$       & K  & Potential temperature \\
$CW$             & $\nicefrac{kg}{kg}$ & Cloud water          \\
$CL$             & \% & 3D Cloud cover            \\
$cnvuf$          & $\nicefrac{kg}{sm^2}$ & Convective updraft flux \\
$cnvdf$          & $\nicefrac{kg}{sm^2}$ & Convective downdraft flux \\
$PR$             & mm & Precipitation         \\
\hline
\multicolumn{3}{l}{2D fields - for Surface} \\
$PS$             & hPa & Surface pressure                     \\
$T2$          & K  & Temperature at 2m height               \\
$Rh2$             & \% & Relative humidity at 2m height \\
SH              &  $\nicefrac{W}{m^{2}}$ & Surface flux of sensible heat \\
LH             & $\nicefrac{W}{m^{2}}$ & Surface flux of latent heat \\
$\tau$         & $\nicefrac{N}{m^{2}}$ & Surface stress               \\
SST            & K & Sea surface temperature \\
SWC            & $\nicefrac{m^{3}}{m^{3}}$ & Soil water content      \\
%DSWC           & m$^3$/m$^{3}$ & Deep soil water content$^\ast$   \\
lspr             & m & Large scale precipitation \\
cpr              & m & Convective precipitation \\
sdepth         & m & Snow depth \\
ice            & \% & Fraction of ice \\  
$SMI1$           &    & Soil moisture index level 1 \\
$SMI3$           &    & Soil moisture index level 3 \\
u10/v10        & m/s  & wind at 10 m height \\
\hline
\end{tabular}\\
%Notes: $\ast$ not yet used
\end{center}
\end{table}


\subsection{New Emission System}

The model is now more flexible to use different type of emission data such as 'ascii' and 'netcdf' format.  The emission data is now linked via the 'config\_emep.nml' file.  There are two types of 'netcdf' data tested.  

First one is:  All data are stored in a single netCDF file, i.e., the country code and emission data of all components for all sectors are given in one file with the fraction of data covered by each grid cell at the boundaries. Thus the data set contains 80 2-dimensional variables - 7 components in 11 sectors, which are represented as poll\_secNN where 'NN' is the sector number, number of countries shared by each grid cell 'Ncodes', which will be more than one at the country borders wich maximum of 5 countries sharing a border, and map factors for i and j - and 78 3-dimensional variables which are the country codes 'Codes' and the fraction of data covered by each grid cell in each country 'fractions\_poll\_secNN'.  Similar to ascii data, the ship emission data is distributed in grids together with the country code.  The description of the variables in the datset is given in the Table~\ref{Tab:Emisdata}: 

The second netCDF data type contains data for Global domain.  This data set contains one data file for each component similar to ascii data, i.e., 7 data sets in netCDF format, i.e., 'Emis\_poll.nc'.  But the data is stored in each file as 'country per sector for component', i.e., 'Emis:CC:snap:NN' where 'CC' is the country code and 'NN' is the snap sector number .  Thus data having only positive emissions of sectors are included in each data file and hence the number for components in different files can vary.  Ship emissions are not included in this data set.  So we can link the ship emissions to the model using 'config\_emep.nml' when using this data.  

The flexibility in using any type of emission data is made possible in 'config\_emep.nml'.  Compared to previous model version which used the emission flag in both 'ModelConstants\_ml.f90' and 'config\_emep.nml' - 'EMIS\_SOURCE=emislist' - the new setup uses 'EMIS\_SOURCE=Mixed' in both the module and namelist file.  The link to input data should be specified in namelist using the flag 'EmisDir' or 'DataDir', no other names are recognised by the model.  Also we can choose to include or exclude the emissions from a specific country or region is also possible. The regions given in the 'config\_emep.nml' are specified in the 'Country\_ml.f90' file.  Similarly you can define a new region of your choice in 'Country\_ml.f90' and call it from the namelist.   A piece of 'config\_emep.nml' is given below and the explanation of this follows thereafter.  


\begin{center}
\begin{table}
\caption[fracemi]{Variable description of netCDF Emission data including Fractions
\label{Tab:Emisdata}}
%\begin{center}
%\begin{small}
% \hspace{-1cm}
\begin{tabular}{lll}
\hline


{\bf Variable name } & {\bf Description}\\

Ncodes & No of countries shared by each grid cell\\
poll\_secNN & Component at each sector \\
Codes & Country code number \\
fractions\_poll\_secNN & Fraction of data covered by grid cell\\

\hline
\end{tabular}
%\end{small}
%\end{center}
\end{table}
\end{center}



\begin{quote}
\begin{verbatim}

   EMIS\_SOURCE                = 'Mixed',
   EMIS\_TEST                  = 'None', 
   EmisDir                     = 'input',     !
   emis\_inputlist(1)\%name    = 'EmisDir/emislist', !example of ASCII type

!  emis\_inputlist(1)\%name    = 'EmisDir/Emis\_TNO7.nc', ! Fractions type
!  emis\_inputlist(1)\%name    = 'EmisDir/Emis\_POLL.nc', ! netCDF data 
                                                          ! per component
!  emis_inputlist(1)\%incl(1:) = 'EUMACC2',               ! include countries  
                                                          ! only from 'EUMACC2' list
!  emis_inputlist(1)\%excl(1:) = 'INTSHIPS',              ! exclude ship emissions 
                                                          ! when sea areas are 
							  						      ! not divided
! 
\end{verbatim}
\end{quote}

The above piece of code from 'config\_emep.nml' shows that the emissions source flag used is now 'Mixed'.  Emission data is stored under 'EmisDir'.  The following lines with 'emis\_inpulist(1)\%name' links to the emission data that is to be used.'emis\_inputlist(1)\%incl(1:)' in the above example shows that we are using the emissions only from 'EUMACC2' region which is defined in 'Country\_ml.f90' that includes 41 countries including EU28, EEA, Switzerland, GUAM and some FarEast European countries.  'emis\_inputlist(1)\%excl(1:)' shows that the internal ship emissions are exluded.  In this set up of choosing emissions data one can use either 'include' or 'exclude' flag at the moment, using both at the same time is not implemented yet.  


\subsection{Global Ozone}
Initial concentration of ozone are required in order to
initialize the model runs. Boundary conditions along the sides of the model
domain and at the top of the domain are then required as the model is
running.

The Logan\_P.nc file contains monthly averaged fields in netCDF format. 
The initial and background
concentrations are based on the Logan (1998) climatology. The Logan
climatology is scaled by Unimod according to the Mace Head measurements as
described in Simpson {\sl et al.} (2003). For a number of other species, 
background/initial conditions are set within the model using functions 
based on observations (Simpson {\sl et al.}, 2003 and Fagerli {\sl et al.}, 2004).

\subsection{BVOC emissions}
Biogenic emissions of isoprene and monoterpene are calculated in the
model as a function of temperature and solar radiation, using the landuse
datasets. The light and temperature depencies are similar to those
used in the original open source model, see 
Chapter 4.2 of the EMEP Status Report 1/2003 Part I (Simpson
{\sl et al.}, 2003).

Biogenic VOC emission potentials (i.e. rates at 30$^\circ$C and full sunlight)
are included for four different forest types in the netCDF file 
EMEP\_EuroBVOC.nc. These emission potentials have unit $\mu g/m^{2} /h$, and
refer to emissions per area of the appropriate forest category. In 
addition, default emission potentials are given for other
land-cover categories in the file Inputs\_LandDefs.csv. 
The underlying emission potentials, land-cover data bases, and model
coding have however changed substantially since model version v.2011-06. The new approach
is documented in Simpson {\sl et al.}, 2012.

\subsection{Landuse}

Landuse data are required for modeling boundary layer processes
(i.e. dry deposition, turbulent diffusion).
The EMEP/MSC-W model can accept landuse data from any
data set covering the whole of the domain, providing reasonable 
resolution of the vegetation categories. Gridded data sets providing
these landuse categories across the EMEP domain have been created
based on the data from the Stockholm Environment Institute at York 
(SEI-Y) and from the Coordinating Center for Effects (CCE). 
16 basic landuse classes have been identified for use in the
deposition module in the model, and three additional ``fake'' landuse
classes are used for providing results for integrated assessment
modeling and effects work.

There are two netcdf files included, one file ``Landuse\_PS\_5km\_LC.nc`'' on 5 km resolution over the EMEP domain, and a global ``LanduseGLC.nc''. The different landuse types are desribed in Simpson et al (2012). 

% There is a {\bf header} in the file that contains short abbreviations 
% for the different land cover
% types, e.g. CF for temperate/boreal coniferous forest. The landuse
% types are summarized in Table 5.1 in Chapter 5 of the EMEP Status
% Report 1/2003 Part I (Simpson {\sl et al.}, 2003).
% 
% The different landuse types are given as a percentage of area for each 
% EMEP grid cell in the {\bf ASCII file} ``Inputs.Landuse''. 

\subsection{Degree-day factor}
Domestic combustion which contribute to a large part of SNAP-2, varies on the daily 
mean temperature. The variation is based on the heating degree-day concept. These 
degree days are pre-calculated for each day and stored in the file DegreeDayFactors.nc. 
See Simpson et al. (2012) section 6.1.2. 


\subsection{NO$_x$ depositions}
Areas with high NO deposition loads have greater soil-NO emissions. To include this in 
the model, a netcdf file where pre-calculated N-depositions are included. The file made by 
the results from the EMEP/MSC-W model runs over a 5-year period. 


\subsection{Road Dust}
Road traffic produces dust. These emissions are handled in the EMEP/MSC-W model in the 
{\bf Emissions\_ml.f90} module. To include road dust, set USE\_ROADDUST = .true. in ``config\_emep.nml''. There are two files included in input data, RoadMap.nc and 
AVG\_SMI\_2005-2010.nc. RoadMap.nc include gridded roads and PM emissions over Europe, 
AVG\_SMI\_2005-2010.nc are global. 

\subsection{Aircraft emissions}
In the EMEP/MSC-W model aircraft emissions are 'OFF' by default. 
They can be switched 'ON' by setting USE\_AIRCRAFT\_EMIS = .true. in ``config\_emep.nml'' and download the data from \url{http://www.pa.op.dlr.de/quantify}.  
The EMEP model uses data provided by the EU-Framework Programme 6 Integrated 
Project QUANTIFY (\url{http://www.pa.op.dlr.de/quantify}). However, before using 
these data a protocol has to be signed, which is why the data file can not be provided 
directly on the EMEP/MSC-W Open Source website. If you want to use aircraft emissions go to 
\url{http://www.pa.op.dlr.de/quantify}, click on 'QUANTIFY emission inventories and scenarios', 
and then click on 'Register'. That page will provide information about the registration 
process and the protocol that has to be signed. Once you are registered, click 'Login' and 
provide user name and password. On the new page, search for 'Emissions for EMEP', which 
links directly to the Readme file and the emission data file in netCDF format. Download the 
emission data file and place it in the input folder.

\subsection{Surface Pressure}

If USE\_AIRCRAFT\_EMIS = .true. in { \bf config\_emep.nml,} then in addition to the Aircraft 
Emission file, there will be need for a SurfacePressure.nc file, which is already in the /input folder. 
The netCDF file consists of surface pressure fields for each of the months in 2008 called surface\_pressure, 
and one field for the whole year called surface\_pressure\_year. All fields are given in Pa. 

\subsection{Forest Fire}
As of model version rv3.9 (November 2011), daily emissions from forest and vegetation fires are taken from the “Fire INventory from NCAR version 1.0” (FINNv1,
Wiedinmyer et al. 2011). Data are available from 2005, with daily resolution, on a fine 1 km×1 km grid. We store these data on a slightly coarser grid (0.2$^\circ$×0.2$^\circ$) globally for access by the EMEP/MSC-W model. To include forest fire emissions set 
USE\_FOREST\_FIRES = .true. in ``config\_emep.nml'' and download the 
2012 GEOS-chem daily data \url{http://bai.acd.ucar.edu/Data/fire/}. The data needs to be stored with units mole/day in a netCDF file called FINN\_ForestFireEmis\_2012.nc 
compatible with the { \bf ForestFire\_ml.f90 } module. 

\subsection{Dust files}???????

The annual ascii data for sand and clay frations as well as the monthly data for boundary and initial conditions for dust from Sahara are replaced with a single netCDF file 'Soil\_Tegen.nc' since 2013.   This covers data for a global domain in 0.5X0.5 degree resolution.  

The variables 'sand' and 'clay' gives the fraction (in \%)  
of sand an clay in the soil for each grid cell over land. (((The files are provided by 
Maxilian Posch, Netherlands Environmental Assessment Agency, CCE, personal communication????? .)) 


The files are used by the module {\bf DustProd\_ml.f90 }, which calculates windblown dust 
emissions from soil erosion. Note that the parametrization is still in the development and 
testing phase, and is by default 'turned off'. To include it in the model calculations, set 
USE\_DUST = .true. in ``config\_emep.nml''.
The user is recommended to read carefully documentation and
comments in the module {\bf DustProd\_ml.f90}.

There is also a possibility to include boundary and initial conditions for dust from Sahara. 
The input file gives monthly dust mixing 
ratios (MM - month, e.g. 01, 02, 03,...) for fine and coarse dust from Sahara. The files are based on calculations 
from a global CTM at the University of Oslo for 2000. 
To include Saharan dust, set USE\_SAHARA = .true. in ``config\_emep.nml''.

Another source for dust is an arid surface. This is accountet for by soilmosture calculations in {\bf DustProd\_ml.f90 }. 
Together with Soil Water Index from the meteorology files and permanent wilting point (pwp) from SoilTypes\_IFS.nc. 
This file is global and netcdf. See Simpson et al. (2012) section 6.10.



\section{ASCII files}


% 
% 
% 
% \subsection{Land/Sea mask}
% This file, based on the meteorological model's roughness length, is used
% to assign a land/sea mask within the Unified EMEP model, since there is
% a need to modify the stability information for coastal grid cells
% (which contain land, but is not resolved by the NWP model supplying 
% meteorological data to the Unified EMEP model). 
% 
% The gridded {\bf ASCII} file ``landsea\_mask.dat'' contains 3 columns. 
% The first two columns represent the `i' and `j' indices of the EECCA
% grid and the third column gives the class of the grid cell, where the 
% EMEP model use 0 for ocean, and $\geq$ 1 as land.



\subsection{Natural $SO_2$}
Natural $SO_2$ emissions (dimethylsulfide (DMS) from sea) are provided 
as monthly gridded files.  
The values are given at the surface in $\mu$g/m$^2$ for each grid cell in the domain. 

\subsection{Volcanoes}

Emissions from volcanoes are included for Italy, based upon the
officially submitted data.
To consider the volcanic emissions, we need to feed the location
and height of volcanoes into the model. 

The input file ``VolcanoesLL.dat'' in {\bf ASCII} format contains the input data for the two active volcanoes
 Enta and Stromboli. The two first columns gives their latitudes and longitudes, their height as sigma level
 in column 3 and their emission in ktons per year in column 4. 

\subsection{Gridded Emissions}
The emission input for the EMEP/MSC-W model consists of gridded
annual national emissions based on emission data reported every year
to EMEP/MSC-W (until 2005) 
and to CEIP (from 2006) by each
participating country. 
More details about the emission input with references can be
found in Chapter 4 of the EMEP Status Report 1/2003 Part I 
(Simpson et al., 2003).

The 7 gridded emission input files ``emislist.poll'' are available in 
{\bf ASCII} format for the following compounds: CO, NH$_{3}$,
NO$_{x}$, PM$_{2.5}$, PM$_{co}$, SO$_{x}$ and VOC.

The gridded emission files contain 16 columns where the first column 
represents the country code
(\url{http://www.emep.int/grid/country_numbers.txt}), 
the second and the third columns are the `i' and `j' indices of the
EMEP grid, the fourth and fifth columns are the total emissions from
low and high sources, and the last 11 columns contain emissions from 
10 anthropogenic SNAP sectors 
(\url{http://reports.eea.eu.int/technical_report_2001_3/en}) and 1 
source-sector called``Other sources and sinks'', which include natural and
biogenic emission sources. The data are given with the unit: $Mg$.

{\bf Acknowledgement:} EMEP

\subsection{Time factors for emissions}

Monthly and daily time factors for emission of the 7 compounds 
(CO, NH$_{3}$, NO$_{x}$, PM$_{2.5}$, PM$_{co}$, SO$_{x}$ and VOC). 
There is one file available per compound in {\bf ASCII} format. 

The first two columns in the files represent the country code \\
(\url{http://www.emep.int/grid/country_numbers.txt}), the second column 
represents the sector (\url{http://webdab.emep.int/sectors.html}). In the monthly files, 
the 12 consecutive columns represent the time factors corresponding to 
the months of the year. In the daily files there are 7 consecutive columns representing 
the time factor for each day of the week. 

The file HOURLY-FACS 
includes factors for each of the eleven SNAP sectors for every hour (the columns) for 
each day of the week, see Simpson et al. (2012) section 6.1.2

% \COMMENT{I moved all emission-related factors here}

\subsection{Emission heights}
A vertical distribution for the elleven SNAP sectors are given in the file EmisHeights.txt. 
The file has seven vertical levels, over the collumns and the SNAP sectors given in the first row. 
Read more in Simpson et al. (2012) section 6.1.1.

\subsection{Emission factor for scenario runs}\label{sec:femis}
Scenario run in the case of the EMEP/MSC-W model means a run to test
the impact of one or more pollutants from a particular
country. 

Emission factors are applied to specified countries and
emission sectors and can be set by changing the {\bf ASCII} file
``femis.dat''. 
This file can be changed by the users according to their needs.

The file contains several columns (the number is flexible). The first column represents
the country code (\url{http://www.emep.int/grid/country_numbers.txt}),
the second represents the sector
(\url{http://reports.eea.eu.int/technical_report_2001_3/en}) 
where `0' means all sectors, and then in the remaining
 columns one can specify
which emissions to reduce. Here 1.0 means no reduction of the given
pollutant 
(sox/nox/voc/nh3/co/pm25/pmco) from sectors of specified country. The
number following the first text (``Name'') in line 1 (number 5 in
the downloaded file) gives the number of pollutants treated in the file.


\subsection{Chemical speciation of emissions}

%\NEW{
Many of the emission files give emissions of a group of compounds, e.g.
NOx includes NO+NO$_2$, and VOC can include many compounds. The information
needed to retreive emissions of individual compounds from these the
gridded files is given in  files labelled emissplit.defaults.$poll$ or
emissplit.specials.$poll$, where $poll$ can be nox, voc, etc.

The defaults file give the emission split for each SNAP sector (
one per row, with second index being the SNAP sector), which
is applied to all countries by default. For VOC this split
was derived from the UK inventory of Passant (2002),
as part of the chemical comparison project of Hayman {\sl et al.} (2011).

The specials files are in general optional, and can be used to specify
speciation for particular countries or SNAP sectors. The
1${^st}$ column specifies the country code of interest, the second the SNAP sector. 

If forest fires are used, then the file emissplit.specials.voc is required
(not optional), and the country-code 101 used to specify the VOC speciation
of forest fires in this file.
% }

\subsection{Lightning emissions}
Emissions of NO$_{x}$ from lightning are included in the model
as monthly averages on a T21 (5.65$^{\circ}\times$5.65$^{\circ}$) resolution (K{\o}hler {\sl et al.}, 1995). 
The lightning emissions are defined on a 64$\times$32 grid with 17 vertical
levels, with global coverage, and are provided as 12 {\bf ASCII} files
``lightningMM.dat''.

\subsection{Landuse definitions}
For the vegetative landuse categories where stomatal modeling is
undertaken, the start and end of the growing season (SGS, EGS, in days) must be specified. 
The calculation of SGS and EGS with respect to latitude is done 
in the module {\bf LandDefs\_ml.f90}. 
The parameters needed to specify the
development of the leaf area index (LAI) within the growing season
are given in the ASCII file ``Inputs\_LandDefs.csv''. 
For more information, see chapter 5  of the EMEP Status Report 1/2003 Part I (Simpson {\sl et al.}, 2003).


The file, designed to be opened with excel or gnumeric,
contains a header briefly explaining the contents of the 
14 columns. 
The first three columns are representing the landuse name, code (which
are consistent with those in ``Landuse.Input'' file) and
type (grouping of the landuse classes). The fourth column 
% \NEW{
(PFT) gives a plant-functional type code (for future use),
 the fifth gives
the maximum height of vegetation ($m$), the sixth indicates albedo (\%) and
the seventh indicates possible source of NH$_{x}$ (0 off/1 on,
curently not used). Columns 8 to 11
define the growing season (day number), column 12 and 13 lists the
LAI minimum  and maximum ($m^{2}/m^{2}$) and 
columns 14 and 15
defines the length of the LAI increase and decline periods (no. of days).
% \NEW{
Finally, the last four columns give default values of 
foliar biomass and biogenic VOC emission potentials. See Simpson
et al., (2012) for details.
% }


\subsection{Stomatal conductance}
Parameters for the stomatal conductance model, deposition of O$_{3}$ and
stomatal exchange (DO3SE) must be specified. That are based upon the ideas in
Emberson {\sl et al.}, 2000, and are discussed in Simpson and Emberson,
2006 and Tuovinen et al. 2004.

The  ASCII file ``Inputs\_DO3SE.csv'' provides land-phenology data
of each landuse type for stomatal conductance calculations. The 
data are summarised in Table 5.1 in Chapter 5 of the EMEP 
Status Report 1/2003 Part I (Simpson {\sl et al.}, 2003).

The file contains a {\bf header} with the contents of the file,
with different factors needed for each of the landuse classes used
in the EMEP/MSC-W model. The first two columns represent the
landuse code (which are consistent with those in ``Landuse.Input'' file)
and name. The next 22 values are different phenology factors.


\subsection{Photo-dissociation rates}
The photo-dissociation rates (J-values) are provided as lookup
tables. The method is previously described in Jonson {\sl et
al.}, (2001). J-values are provided as clear sky, light cloud and dense
cloud conditions, and the model interpolates between these according
to cloudiness from the meteorological input data. In the lookup tables
data are listed for every 10 degree latitude at an interval of 1
degree zenith angle at every model height.

For the two types of cloud conditions there are one {\bf ASCII} file 
averaged for each season (SS); 01, 02, 03 and 04. 
For light cloud the four seasonal files are called ``jcl1kmSS.dat'', for dense cloud conditions
the four seasonal files are called ``jcl3kmSS.dat'', and then for clear sky four files 
called ``jclearSS.dat''. In addittion there are two files for june called jcl1.jun 
and jcl3.jun.

Each file contains 18 columns. The first column is latitude of zenith 
angle and then the
next 17 are the values for the model levels with the unit: $1/s$. 
For more details about these rates, please read Chapter 7.2 of the EMEP
Status Report 1/2003 Part I (Simpson {\sl et al.}, 2003).


\subsection{Site and Sonde locations for output}\label{sec:sitessondes_input}
The model provides a possibility for extra output data of surface concentration 
for a set of specified measurement site locations and concentrations for the vertical 
column above a set of specified locations. These site and sonde locations are listed 
in the {\bf ASCII} files ``sites.dat`` and ``sondes.dat`` files. These files can be 
changed by the user, this is described in section \ref{sec:sitesonde}.

