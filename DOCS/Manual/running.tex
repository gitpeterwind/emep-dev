\chapter{Submitting a Run}

Once you have made necessary changes with the code and checked
everything with the input files go to the 'run' directory. \\

\textbf{cd run} \\
\\


Learn how to use ``run.pl'' (or grun.pl) - the perl scripts which generates
the appropriate Makefile and fortran files. run.pl allows
the user to change time-period, domain, inputs, etc., without
having to change any fortran files by hand. The procedure for running
this is somehwat different on the Linux cluster and  Origin (gridur)
machines.

Check once again all the meteorological and emission data are in place.  Change
the user name to your corresponding username.  Set the output path
properly. 


An extract from run.pl  is given below.  Set the number of CPUs you
want to use in the beginning of this script. 


In the following example, you are using 32 CPUs which is specified
with the command \\

\textbf{PBS -lnodes=32} \\
\\

The script is very much self explanatory.  Set the length of the run
using any of the following options.  If it is a specific number of months it
is better to set this with \\

\textbf{mm1} and \textbf{mm2}\\


If it is a few days run, NTERM is calculated manually and write the
number of timesteps in 'NTERM\_CALC'.  

For example \textbf{NTERM\_CALC=25} means a run of three days. 

 The example given here is for a run of 3 months.  

\begin{verbatim}
#!/usr/local/bin/perl
#Queue system commands start with #PBS (these are not comments!)
# lnodes= number of nodes, ppn=processor per node (max4)
#PBS -lnodes=32
# wall time limit of run
#PBS -lwalltime=01:00:00
# lpmeme=memory to reserve per processor (max 4 or 16GB per node)
#PBS -lpmem=200MB
# account for billing
#PBS -A nn2890k
#
$year = "2000";        # Meteorology and emissions year

my $SR = 0;            #NEW Set to 1 for source-receptor stuff

# iyr_trend:
# :can be set to meteorology year or arbitrary year, say 2050

$iyr_trend = $year;  
$iyr_trend = "2010" if $SR ;  # 2010 assumed for SR runs here

print "Year is $yy YEAR $year Trend year $ir_trend\n";

if ( $year == 2000 ) {
  $MetDir = "/work/mifads/metdata/$year" ;
...
}
#---  User-specific directories (changeable)

$DAVE        = "/home/u2/mifads";      
$PETER       = "/home/u4/mifapw";      

$USER        =  $DAVE ;      

#ds - simplified treatment of BCs and emissions:

$OZONE = 1, $ACID = 0;     # Specify model type here

$version     = "Unimod" ;  
$Split       = "BASE_MAR2004" ;               #  -- Scenario label for MACH - DS
$ProgDir     = "$USER/Unify/Unimod.$testv";   # input of source-code
$MyDataDir   = "$USER/Unify/MyData";          # for each user's femis, etc.
$DaveDataDir = "$DAVE/Unify/MyData";          # for each user's femis, etc.
$DataDir     = "$DAVE/Unify/Data";      # common files, e.g. ukdep_biomass.dat
$PROGRAM     = "$ProgDir/$version";         # programme
$WORKDIR     = "$WORK{$USER}/Unimod.$testv.$year";    # working directory

......
$emisdir     = "$SVETLANA/Unify/MyData/emission/2004_emis00_V2";

# Specify small domain if required. 
#                 x0   x1  y0   y1
@largedomain = (   1, 170,  1, 133 ) ;
@smalldomain = (  18, 169,  7, 124 ) ;     # OSPAR/HELCOM domain+border-south

$RESET        = 0   ;  # usually 0 (false) is ok, but set to 1 for full restart
$COMPILE_ONLY = 0   ;  # usually 0 (false) is ok, but set to 1 for compile-only
$INTERACTIVE  = 0   ;  # usually 0 (false), but set to 1 to make program stop
                       # just before execution - so code can be run interactivel.

$NDX   = 8;           # Processors in x-direction
$NDY   = 4;           # Processors in y-direction
if ( $INTERACTIVE ) { $NDX = $NDY = 1 };

$mm1   =  1;       # first month
$mm2   =  3 ;      # last month

$NTERM_CALC =  calc_nterm($mm1,$mm2);

$NTERM =   $NTERM_CALC;    # sets NTERM for whole time-period
  # -- or --
# $NTERM = 2;       # for testing, simply reset here

  print "NTERM_CALC = $NTERM_CALC, Used NTERM = $NTERM\n";

# <---------- end of normal use section ---------------------->
\end{verbatim}


Once you have verified all these and set everything properly, submit
the run using the command \\

\textbf{qsub run.pl}\\
\\

\textbf{That's it!!!  You have already submitted a run.}  This script
does the compilation of the code and submission of the run as well.    

Check the status of the run by typing the command\\

\textbf{qstat -a}\\

This will give you a list of jobs that have been submitted on the
machine. An example is given below.  In this, the machine is called
'snowstorm.uit.no'.  The columns are explained in the first line.   

\begin{verbatim}

snowstorm.uit.no:
                                                      Req'd  Req'd   Elap
Job ID  Username Queue    Jobname    SessID NDS   TSK Memory Time  S Time
------- -------- -------- ---------- ------ ----- --- ------ ----- - -----
63277   conradj  itanium  PdO_OH_Cla  31936    24  --    --  300:0 R 189:2
65922   bour     itanium  go_Abase     7589     1  --    --  100:5 R 78:36

\end{verbatim}


If you want to cancel a run use the command \\

\textbf{qdel jobid}\\



Once the run is finished you will have the following files in your
'run' directory. 

\textbf{run.pl.e$<$jobid$>$}\\
\textbf{run.pl.o$<$jobid$>$}\\

The first file with '.e' extension is the error file and a successful
completion of the run will result an empty error file.  The second
file with '.o' extension is the object file and this file gives all the
details about the run including how many CPU hours it has taken etc.  

If the run crashed due to any reason, check the error file which will
lead you to the reason for the crash.  Fix the bug and submit the run
again.  That's it!!!  


