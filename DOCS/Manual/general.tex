\chapter{Welcome to EMEP }

This guide gives a brief documentation of the EMEP unified model.
It is intended primarily as a guide as to how to run the model, and
to help users wishing to understand or change 
the model in terms of domains, outputs, chemistry, etc.
This guide is also intended to guide new programming with a few
simple ideas and rules.

The main documentation for the model is given in:

\begin{itemize}
\item
D.~Simpson, H.~Fagerli, J.E. Jonson, S.~Tsyro, P.~Wind, and J.-P. Tuovinen.
{The EMEP Unified Eulerian Model. Model Description}.
EMEP MSC-W Report 1/2003, The Norwegian
Meteorological Institute, Oslo, Norway, 2003.
\item
Fagerli, H., Simpson, D., and Tsyro, S., {Unified EMEP model: Updates}, in {\em
  EMEP Report 1/2004, Transboundary acidification, eutrophication and ground
  level ozone in Europe. Status Report 1/2004\/}, pp. 11--18, The Norwegian
  Meteorological Institute, Oslo, Norway, 2004.
\end{itemize}

Copies of both reports can be obtained from \url{www.emep.int}. An HTML
copy of EMEP report 1/2003 is available at \url{www.emep.int/UniDoc/index.html}.


\section{Computer Information}

To compile the EMEP model you need:

\textbf{Fortran 90 compiler}
\textbf{NetCDF Library}
\textbf{MPI Library}

It is necessary to compile with double precision reals (8 bytes
reals).  The program has been tested with Intel compiler on an
Itanium2 cluster, and XL compiler on power5+ cluster; example of
Makefiles are provided. Running on different systems or with different
compilers is not expected to lead to any problem.

The code has been tested with from 1 to 128 CPU, and scales well.  If
only one CPU is used 1-2 GB memory is required and if more than one
for example 32 CPU are used, 200 MB of memory per CPU is enough (in
the case of a 170X133 grid size).

\section{Getting Started}

A tar file of the model source code can be downloaded from the web
page under the section 'Download'.   Once the model is downloaded,
untar the zipped tar file using  the command \\

\textbf{tar -xzvf emep.tar.gz} \\

This will give you a directory called 'Unimod'.  Go to this directory
by typing 'cd Unimod' and do a listing and you will see the following
\\

\textbf{cd Unimod} \\
          \textbf{ls -lt}


\begin{verbatim}
-rwxr-xr-x  2 usr emep   4096 May 11 13:23 work  Makefile \\
-rwxr-xr-x  2 usr emep   4096 May 11 13:23 work  Log.Changes \\
-rwxr-xr-x  2 usr emep   4096 May 11 13:23 work  README \\
drwxr-xr-x  2 usr emep   4096 May 11 13:23 work  src \\
                - ....f90
drwxr-xr-x  2 usr emep   4096 May 11 13:23 work  doc  \\
                - ....tex \\
                - ....eps \\
drwxr-xr-x  2 usr emep   4096 May 11 13:23 work  run \\
                - ....pl  \\

\end{verbatim}

Read the 'README' first.  This has a brief note about the directory
contents.  Makefile is the main Makefile for the model.  Each time
when the model code is updated or changed due to any reason, it is
documented in a Log file while committing the changes to CVS and this
file is stored in 'Log.Changes' file.  The src directory contains all
the main .f90 files of the model. 'doc' contains the latex code of the
'cookbook'.  Finally, the run scripts are stored under 'run'.  These
are perl scripts and a detailed description about how to handle them
is given in Chapter 4, 'Submitting a Run'.  

Each module and subroutine begins with a small documentation of the
same (and are updated regularly). Next chapter will give a detailed
insight into the necessary input files and their structure.    




