\chapter{Introduction}

This guide gives a brief documentation of the EMEP unified model.
It is intended to complement the CookBook, with a more
in-depth guide for users 
wishing to understand or change 
the model in terms of domains, outputs, chemistry, etc.
This guide is also intended to guide new programming with a few
simple ideas and rules.

The main documentation for the model is given in:

\begin{itemize}
\item
D.~Simpson, H.~Fagerli, J.E. Jonson, S.~Tsyro, P.~Wind, and J.-P. Tuovinen.
{The EMEP Unified Eulerian Model. Model Description}.
EMEP MSC-W Report 1/2003, The Norwegian
Meteorological Institute, Oslo, Norway, 2003.
\item
Fagerli, H., Simpson, D., and Tsyro, S., {Unified EMEP model: Updates}, in {\em
  EMEP Report 1/2004, Transboundary acidification, eutrophication and ground
  level ozone in Europe. Status Report 1/2004\/}, pp. 11--18, The Norwegian
  Meteorological Institute, Oslo, Norway, 2004.
\end{itemize}

Copies of both reports can be obtained from \url{www.emep.int}. An HTML
copy of EMEP report 1/2003 is available at \url{www.emep.int/UniDoc/index.html}.




The main documentation for each subroutine should be contained within
each subroutine (and updated regularly!).

\section{First steps}

Learn how to use ``run.pl'' (or grun.pl) - the perl scripts which generates
the appropriate Makefile and fortran files. run.pl allows
the user to change time-period, domain, inputs, etc., without
having to change any fortran files by hand. The procedure for running
this is somehwat different on the Linux cluster and  Origin (gridur)
machines.

An extract from run.pl:
\begin{verbatim}
#!/usr/local/bin/perl
#
# To submit on GRIDUR with say 32 processors, put grun.pl into work directory, 
# and from there do something like:
#
#    bsub -n 32 -o log.out < grun.pl
.....

$year = "2000";        # Meteorology and emissions year

my $SR = 0;            #NEW Set to 1 for source-receptor stuff

# iyr_trend:
# :can be set to meteorology year or arbitrary year, say 2050

$iyr_trend = $year;  
$iyr_trend = "2010" if $SR ;  # 2010 assumed for SR runs here

print "Year is $yy YEAR $year Trend year $ir_trend\n";

if ( $year == 2000 ) {
  $MetDir = "/work/mifads/metdata/$year" ;
...
}

#---  User-specific directories (changeable)

$DAVE        = "/home/u2/mifads";      
$PETER       = "/home/u4/mifapw";      

$USER        =  $DAVE ;      

#ds - simplified treatment of BCs and emissions:

$OZONE = 1, $ACID = 0;     # Specify model type here

$version     = "Unimod" ;  
$Split       = "BASE_MAR2004" ;               #  -- Scenario label for MACH - DS
$ProgDir     = "$USER/Unify/Unimod.$testv";   # input of source-code
$MyDataDir   = "$USER/Unify/MyData";          # for each user's femis, etc.
$DaveDataDir = "$DAVE/Unify/MyData";          # for each user's femis, etc.
$DataDir     = "$DAVE/Unify/Data";      # common files, e.g. ukdep_biomass.dat
$PROGRAM     = "$ProgDir/$version";         # programme
$WORKDIR     = "$WORK{$USER}/Unimod.$testv.$year";    # working directory

......
$emisdir     = "$SVETLANA/Unify/MyData/emission/2004_emis00_V2";

# Specify small domain if required. 
#                 x0   x1  y0   y1
@largedomain = (   1, 171,  1, 133 ) ;
@smalldomain = (  18, 169,  7, 124 ) ;     # OSPAR/HELCOM domain+border-south

$RESET        = 0   ;  # usually 0 (false) is ok, but set to 1 for full restart
$COMPILE_ONLY = 0   ;  # usually 0 (false) is ok, but set to 1 for compile-only
$INTERACTIVE  = 0   ;  # usually 0 (false), but set to 1 to make program stop
                       # just before execution - so code can be run interactivel.

$NDX   = 8;           # Processors in x-direction
$NDY   = 4;           # Processors in y-direction
if ( $INTERACTIVE ) { $NDX = $NDY = 1 };

$mm1   =  1;       # first month
$mm2   =  3 ;      # last month

$NTERM_CALC =  calc_nterm($mm1,$mm2);

$NTERM =   $NTERM_CALC;    # sets NTERM for whole time-period
  # -- or --
 $NTERM = 2;       # for testing, simply reset here

  print "NTERM_CALC = $NTERM_CALC, Used NTERM = $NTERM\n";

# <---------- end of normal use section ---------------------->
\end{verbatim}
%OLDIf we let \$hdir be the user's home directory (e.g. Unify/Unimod.rv2_1) and
%OLD\$wdir be the work directory (e.g. /work/mifads/Unimod.rv2_1.1997), then
%OLDwe can submit a job for 8 processors either interactively or as a batch job,
%OLDas illustrated in Table~\ref{RUNPL}.
%OLD
%OLD
%OLD\begin{small}
%OLD\begin{table}
%OLD\begin{tabular}{|llcl|}\hline
%OLD                     &   Cluster    &  & Gridur                \\ \cline{2-2} \cline{4-4}
%OLD  & & & \\
%OLDInteractive          &   cd \$hdir  &  & cd \$wdir              \\
%OLD                     &   run.pl     &  & cp \$hdir/grun.pl .    \\
%OLD                     &              &  & grun.pl               \\
%OLD  & & & \\
%OLDBatch                &   cd \$hdir  &  & cd \$wdir              \\
%OLD                     &              &  & cp \$hdir/grun.pl .    \\
%OLD                     &              & & bsub -n 8 -q express grun.pl \\
%OLD                     &              &  &                        \\ \hline
%OLD\end{tabular}
%OLD\caption{ Submission of run-scripts on cluster and gridur
%OLD  \label{RUNPL}
%OLD}
%OLD\end{table}
%OLD\end{small}
%OLD
%OLD
%OLDThe first option is for interactive use - it will link all files
%OLDand set the model running. For this option a version of the model with
%OLDsmall domain (e.g. 40x40) and using not more than 8 processors is
%OLDrequired (see below).
%OLD
%OLDThe second option is for batch use. Any number of processors
%OLDis possible, although between 8 and 32 will ensure a faster throughput.
%OLDThe number specified here must match that in the run.pl script

\subsection{The *.pat, Make.log files}

The Par\_ml.f90 file needs to be changed when the model domain or
number of processor used is changed. Rather than 
do this by hand, or keep multiple copies  for all
conceivable options, 
the `pattern' file {\bf Par\_ml.pat} is 
used. When run.pl is run, this is converted into the
appropriate  Par\_ml.f90 with the relavent domain and
processor number as specified in the run.pl script. This
methodology is intended to simplify the process of changing 
domains and processors, and ensure consistsent Makefiles
and Par\_ml.f90. 

The values used for domain, and processor number (NDX and NDY)
are stored after compilation in the Make.log file. run.pl
reads this file, and recompiles the programme if the
domain or processor information have changed. 

