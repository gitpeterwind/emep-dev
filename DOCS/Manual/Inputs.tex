\chapter{Input files}

This section describes some standard input files for the
EMEP model. In general, the larger input files
are in netCDF format, and the smaller ones in ASCII.
Table~\ref{Tab:Inputs} lists the files used.

\section{NetCDF files}

The Meteorology and Boundary and Initial conditions files are provided
in NetCDF format.  3 hourly meteorological outputs of PARLAM-PS, the
specially dedicated version of the 
operational model HIRLAM is used as the input
meteorology for EMEP.  These are available as daily files. \\

The boundary and Initial conditions file contain the boundary
conditions for the chemical species.  This monthly file contains the
3-D concentration and background concentrations  of the advected
species.  See 'BoundaryConditions\_ml.f90' routine of the model source
code for more details about this data set.  

\section{ASCII files}

All other files including the emissions, landuse, snowcover etc., are
in ascii format.  All ascii input files contain a few lines header
information of the corresponding file.   \\

Gridded emission files contain 16 columns where the first column
represents the country code, the second and third are the 'i' and 'j'coordinates of the model grid, the fourth and fifth are the total emissions from
low (below 100 m of the model level, mainly from traffic,
agricultural emissions etc.,) and high sources (above 100 m of the
model level, which are from industry, power plants etc.),
and the rest 11 columns contain anthropogenic emissions from 10 SNAP1
emission sectors (Source-Nomenclature for Air Pollution) and natural emissions. Sector means
the source type of emissions from each country and there are 11
sectors defined in EMEP.  A listing of all type
of sectors are given in 'EmisDef\_ml.f90' of the mode source code.   An arbitrary
example file would look like  \\

\textbf{An example emission file:}

\begin{verbatim}

#countrycode i j low high sectors1 2 3 4 5 6 7 8 9 10 11 
1 1 4 20.8 1.0 0.0 7.9 0.6 0.0 0.0 0.0 6.4 4.1 0.8 0.0 0.0
1 1 5 24.0 0.0 2.5 5.7 0.8 0.0 3.0 5.0 4.4 2.1 0.5 0.0 0.0
1 1 7 29.7 0.0 5.5 8.9 0.6 0.0 0.0 0.0 8.4 5.1 1.2 0.0 0.0
..................
etc..........

\end{verbatim}

Landuse data are required in the model, primarily for dry
depositon modelling and for estimation of biogenic emissions.  EMEP
model can accept landuse data from any dataset covering the whole of
the domain and providing reasonable resolution of vegetation
categores.  Gridded data-sets providing these landuse categories
across the EMEP domain have been created based on the data from the
Stockholm Environment Institute ar York (SEI-Y) and form the
Coordinating Centre for Effects (CCE).\\

 16 basic landuse classes have
been identified for the use of deposition module in the model and they
are Temperate/boreal coniferous forests (CF), Temperate/boreal
deciduous forests (DF), Mediterranean needleleaf forests (NF),
Mediterranean broadleaf forests (BF), Temperate Crops (TC),
Mediterranean Crops (MC), Root Crops (RC), Seminatural/Moorland (SNL),
Grassland (GR), Mediterranean Scrub (MS), Wetlands (WE), Tundra (TU),
Desert (DE), Water (W), Ice (I), and Urban (U).  They are listed in
columns following the 'i' and 'j' indexes of the
grids. Input file header explains in detail about all the
columns. \\

Snowcover and Natural SO2 emissions are provided as monthly gridded
files and they are stored in columns of 'i', 'j', and the
'values'. 'Inputs.BVOC' file provide the biogenic VOC emissions.  They
are provided in 4 columns, being the first and second columns as usual
represent the 'i' and 'j' indexes of the model grid and the third and
fourth columns represent the isoprene and terpene emissions in unit of
``$ug/m^{2}/h$''.\\
 

To consider the volcanic emissions, we need to feed the location and
height of volcanoes into the model.  This is provided with the
'Volcanoes.dat' file and the columns in this file represent the 'i'
and j' indexes of the model grid and the third column represent the
model level where the mouth of the volcanoe stands, based on their
height.  The input file contains only two volcanoes which are the only
active volcanoes in Europe.  You might notice that 'Etna, Italy'
appears twice in the input file, which is due to its large base area
it shares two grids in the model.  \\
     

NO$_{x}$ emissions from Aircraft and lightning are also included in the
model. Seasonal commercial aircraft emissions and annual military
aircraft emissions are read in as gridded data from the same file.
The lightning emissions are provided as monthly files for the whole
year.      \\

Time factors for emissions depending on the chemical compound, sector,
and country for day of the week and month of the year are provided with
the 'DailyFac.pollutnat' and 'MonthlyFac.pollutant' files.  Data are
stored in columns of 'i','j', factors corresponding to each day of the
week in 7 consecutive columns in case of DailyFac file and factors
corresponding to each month of the year in 12 consecutive columns in
case of MonthlyFac file.\\  

The model also need a split file for some of the emission files.  For
egs., only one emission file is provided for NO$_{x}$.  The amount of
NO$_{2}$ compared to other NO is determined by the split file.  It is
land and sector dependent.  The same is true for VOC, but
there are more VOC types present. \\ 

Finally the 'femis.dat' file is provided for setting up a 'scenario run'.  See
chapter 4, 'Submitting a run' for the details about this file.  

 
\bigskip

 The input files are designed to read nicely in gnumeric and other spread-
 sheets (excel, oocalc), and can be either space or comma separated.
 Fortran is quite flexible in its reading of separated data, and
would not distinguish between any of:
\begin{verbatim}
2001  7 23  1.2   2.3
2001,7,23,1.2,2.3
2001, 7, 23,   1.2,2.3
2001, 7, 23,   1.2, 2.3
\end{verbatim}

Whereas some spreadsheets get confused if both spaces and commas
are used. It is suggested that files are kept as 
pure comma-separated,
pure space-separated,
or with separation with both comma and space (as in line 4 above).
\bigskip

Lines starting with `:' are for key-value pairs, e.g. : year 2002
The line ending in `\#HEADERS' should contain the headings of each column.
\bigskip

{\bf IMPORTANT:} One line of column headers *must* be provided, and the
 number of headers (excluding commented out ones)  must match the number of data items.
(Any further header lines, e.g. for units, must be commented out, either with
a starting \# or ending \#SKIP).

 All lines starting `\# ' are ignored. The text will show up nicest in
 spread sheets if enclosed in quotation marks, as in line 1 of the example.

\begin{table}[h]
\caption{Gridded Input Files. NNNN: number, MM: month, POLL: pollutant, AA: text}
\label{Tab:Inputs}
\begin{tabular}{lccc}\hline
Data & name (in F90 code) & format\\ \hline
Meteorology&filNNNN&NetCDF\\
Boundary Conditions&Boundary\_and\_Initial\_Conditions.nc&NetCDF\\
%LOGAN BC&&&&not used?\\
Land use&landuse.dat&i j x1-x18 \\ %ds skip first line\\
Snow cover&snowcMM.dat&i j x\\
Emissions&gridPOLL&i j x1-x13\\
Natural SO$_{2}$&natso2MM.dat&i j x\\
BVOC&Inputs.BVOC&i j x1-x6 \\ %ds skip first line\\
Volcanoes&Volcanoes.dat&i j k   \\ %ds comment lines\\
Time series for emissions&MonthlyFac.POLL&EMEP\\
Time series for emissions&DailyFac.POLL&EMEP\\
Lightning&lt21\_nox.datMM&??& \\
Aircraft&amilt42-nox.dat&??&\\
VOC Splits&vocsplit.defaults&ascii \\
NO$_{x}$ Splits& nixsplit.defaults&ascii\\
femis.dat & femis.dat    & ascii  \\
\hline
\end{tabular}
\end{table}

