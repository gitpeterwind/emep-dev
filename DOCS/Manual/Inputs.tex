\chapter{Input files}

This section describes some standard input files for the
EMEP model. In general, the larger input files
are in netCDF format, the smaller ones in an ASCII format.
Table~\ref{Tab:Inputs} lists the files used.

\section{ASCII file formats}


\subsubsection*{An example file:}

%\fbox{
\begin{verbatim}
# "Example of EMEP Input file"
:  Key1 Value1
:  year  2007
:  version rv2_9_8
 mm yy dd v1 v2  #HEADERS
 -  -  - m/s m/s #SKIP 
#DATA:
 02,07, 28,1.2 ,2.3
 02, 07, 29,2.4 ,1.2
 02,07, 30,12.2,6.7
\end{verbatim}
%} % fbox

\bigskip

 The input files are designed to read nicely in gnumeric and other spread-
 sheets (excel, oocalc), and can be either space of comma separated.
 Fortran is quite flexible in its reading of separated data, and
would not distinguish between any of:
\begin{verbatim}
2001  7 23  1.2   2.3
2001,7,23,1.2,2.3
2001, 7, 23,   1.2,2.3
2001, 7, 23,   1.2, 2.3
\end{verbatim}

Whereas some spreadsheets get confused if both spaces and commas
are used. It is suggested that files are kept as 
pure comma-seperated,
pure space-seperated,
or with separation with both comma and space (as in line 4 above).
\bigskip

Lines starting with `:' are for key-value pairs, e.g. : year 2002
The line ending in `\#HEADERS' should contain the headings of each column.
\bigskip

{\bf IMPORTANT:} One line of column headers *must* be provided, and the
 number of headers (excluding commented out ones)  must match the number of data items.
(Any further header lines, e.g. for units, must be commented out, either with
a starting \# or ending \#SKIP).

 All lines starting `\# ' are ignored. The text will show up nicest in
 spread sheets if enclosed in quotation marks, as in line 1 of the example.

\begin{table}[h]
\caption{Gridded Input Files. NNNN: number, MM: month, POLL: pollutant, AA: text}
\label{Tab:GridInputs}
\begin{tabular}{lcccc}\hline
Data & name (in F90 code) & format& comment\\ \hline
Meteorology&filNNNN&special/NetCDF& special treatment \\
Boundary Conditions&POLL.MM&list&\\
%LOGAN BC&&&&not used?\\
Land use&landuse.dat&i j x1-x18& Deposition \\ %dsskip first line\\
Snow cover&snowcMM.dat&i j x& Deposition\\
Emissions&gridPOLL&i j x1-x13&\\
Natural SO2&natso2MM.dat&i j x&\\
Forest&forest.dat&i j x1-x6&for BVOC \\ %ds skip first line\\
Volcanoes&Volcanoes.dat&i j k&   \\ %ds comment lines\\
Lightning&lightnMM.dat&??& \\
Sondes&sondes.dat& Name i j k AA& Output sites\\
Sites&sites.dat  & Name i j k AA& Output sites\\ \hline
\end{tabular}
\end{table}
%In addition aircraft NOx emissions are hardcoded in the Fortran code (AirEmis\_ml.f90).
 
%\section{List of non-gridded input files}

\begin{table}[h]
\caption{Non-grid Input Files.}
\label{Tab:Inputs}
\begin{tabular}{lcccc}\hline
Data & name   &format& comment\\ \hline
Time series for emissions&MonthlyFac.POLL&EMEP&i j x1-x12&\\
Time series for emissions&DailyFac.POLL&EMEP&i j x1-x7&\\
PM Splits                   &pm25split.defaults&& \\
VOC Splits&vocsplit.defaults&& \\
femis.dat           &             & & Modified emissions \\
Landcover parameters& DO3SE\_inputs.csv && Dep. module \\
%Landcover parameters&MM\_gfac1.dat&& Dep. module \\
%Landcover parameters&lde\_gfac2.dat&& Dep. module \\
%Landcover parameters&lde\_biomass.dat&&Dep. module  \\\hline
\hline
\end{tabular}
\end{table}
