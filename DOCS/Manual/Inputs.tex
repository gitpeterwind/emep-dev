\chapter{Input files}

This section describes some standard input files for the
EMEP model. In general, the larger input files
are in netCDF format, and the smaller ones in ASCII.
Table~\ref{Tab:Inputs} lists the files used.

\section{NetCDF file formats}

The Meteorology and Boundary and Initial conditions files are provided
in NetCDF format.  3 hr meteorological outputs of the 
operational model HIRLAM for the year *2000* is used as the input
meteorology for EMEP.  These are daily files. \\

The boundary and Initial conditions file contain the boundary
conditions for the chemical species.  This monthly file contains the
3-D concentration and background concentrations  of the advected
species.  See 'BoundaryConditions\_ml.f90' routine of the model source
code for more details about this data set.  

\section{ASCII file formats}

All other files including the emissions, landuse, snowcover etc., are
in ascii format.  All ascii input files contain a few lines header
information of the corresponding file.   \\

Gridded emission files contain 16 columns where the first column
represents the country code, the second and third are the 'i' and 'j'
indexes of the model grid, the fourth and fifth are the low and high,
and the rest 11 columns represents the emission sectors. An arbitrary
example file would look like  

\textbf{An example file:}

\begin{verbatim}

#countrycode i j low high sectors1 2 3 4 5 6 7 8 9 10 11 
1 1 4 20.8 1.0 0.0 7.9 0.6 0.0 0.0 0.0 6.4 4.1 0.8 0.0 0.0
1 1 5 24.0 0.0 2.5 5.7 0.8 0.0 3.0 5.0 4.4 2.1 0.5 0.0 0.0
1 1 7 29.7 0.0 5.5 8.9 0.6 0.0 0.0 0.0 8.4 5.1 1.2 0.0 0.0
..................
etc..........

\end{verbatim}




\bigskip

 The input files are designed to read nicely in gnumeric and other spread-
 sheets (excel, oocalc), and can be either space or comma separated.
 Fortran is quite flexible in its reading of separated data, and
would not distinguish between any of:
\begin{verbatim}
2001  7 23  1.2   2.3
2001,7,23,1.2,2.3
2001, 7, 23,   1.2,2.3
2001, 7, 23,   1.2, 2.3
\end{verbatim}

Whereas some spreadsheets get confused if both spaces and commas
are used. It is suggested that files are kept as 
pure comma-seperated,
pure space-seperated,
or with separation with both comma and space (as in line 4 above).
\bigskip

Lines starting with `:' are for key-value pairs, e.g. : year 2002
The line ending in `\#HEADERS' should contain the headings of each column.
\bigskip

{\bf IMPORTANT:} One line of column headers *must* be provided, and the
 number of headers (excluding commented out ones)  must match the number of data items.
(Any further header lines, e.g. for units, must be commented out, either with
a starting \# or ending \#SKIP).

 All lines starting `\# ' are ignored. The text will show up nicest in
 spread sheets if enclosed in quotation marks, as in line 1 of the example.

\begin{table}[h]
\caption{Gridded Input Files. NNNN: number, MM: month, POLL: pollutant, AA: text}
\label{Tab:GridInputs}
\begin{tabular}{lcccc}\hline
Data & name (in F90 code) & format& comment\\ \hline
Meteorology&filNNNN&special/NetCDF& special treatment \\
Boundary Conditions&Boundary\_and\_Initial\_Conditions.nc&NetCDF&\\
%LOGAN BC&&&&not used?\\
Land use&landuse.dat&i j x1-x18& Deposition \\ %dsskip first line\\
Snow cover&snowcMM.dat&i j x& Deposition\\
Emissions&gridPOLL&i j x1-x13&\\
Natural SO2&natso2MM.dat&i j x&\\
Forest&forest.dat&i j x1-x6&for BVOC \\ %ds skip first line\\
Volcanoes&Volcanoes.dat&i j k&   \\ %ds comment lines\\
Lightning&lt21\_nox.datMM&??& \\
Sondes&sondes.dat& Name i j k AA& Output sites\\
Sites&sites.dat  & Name i j k AA& Output sites\\ \hline
\end{tabular}
\end{table}
%In addition aircraft NOx emissions are hardcoded in the Fortran code (AirEmis\_ml.f90).
 
%\section{List of non-gridded input files}

\begin{table}[h]
\caption{Non-grid Input Files.}
\label{Tab:Inputs}
\begin{tabular}{lcccc}\hline
Data & name   &format& comment\\ \hline
Time series for emissions&MonthlyFac.POLL&EMEP&i j x1-x12&\\
Time series for emissions&DailyFac.POLL&EMEP&i j x1-x7&\\
PM Splits                   &pm25split.defaults&& \\
VOC Splits&vocsplit.defaults&& \\
femis.dat           & femis.dat    & & Modified emissions \\
Landcover parameters& DO3SE\_inputs.csv && Dep. module \\
%Landcover parameters&MM\_gfac1.dat&& Dep. module \\
%Landcover parameters&lde\_gfac2.dat&& Dep. module \\
%Landcover parameters&lde\_biomass.dat&&Dep. module  \\\hline
\hline
\end{tabular}
\end{table}
