\chapter{Output files}
\label{ch:output}

Output files from a model run are written out in either ASCII, or
(for most data outputs) in netCDF format. 
The different netCDF files are named after the runlabel1 parameter set in 
modrun.sh. 
The model output is written to the same directory as where the runscript 
where submitted, as described in Chapter \ref{ch:SubmitARun}.

To check your model run, already prepared model result files can be 
downloaded from the EMEP/MSC-W Open Source website under 
``Download'' section: ``Model Results''. Unpacked files are placed in
an output directory with model run results for a whole year and sometimes 
with a smaller test run for i.e. April. 

\begin{table}[h!]
\caption[List of model output files]{List of output files written in the
  working directory after a  model run. 
Note: YYYY: year.}\label{tab:output}
% \vspace{0.5cm}
\begin{center}
\hspace{-1cm}
\begin{tabular}{lll}
\hline
{\bf Output data files} &  {\bf Short description} & {\bf Format}\\
\hline
    Base\_day.nc & Gridded daily values of a selection & netCDF\\
&   of compounds.& \\ 
    Base\_hour.nc &Gridded hourly values of a selection &
    netCDF\\  
 &  of compounds.& \\
    Base\_month.nc & Gridded monthly values of a selection&
    netCDF\\
 &  of compounds.& \\
    Base\_fullrun.nc & Gridded yearly values of a selection&
    netCDF\\
 &  of compounds. & \\
    sites\_YYYY.cvs & Surface daily values of a selection&  ASCII\\
 & of stations and compounds.& \\
%    Base.sites.tgz & All the 'sites.MMYY' files in one zipped tar file. & ASCII\\
    sondes\_YYYY.csv & Vertical daily values of a selection& ASCII\\
 &  of stations and compounds. & \\
%    Base.sondes.tgz& All the 'sondes.MMYY' files in one zipped tar file.& ASCII\\ 
& &\\ \hline
{\bf Additional files} &  {\bf Short description} & {\bf Format}\\
    RunLog.out & Summary log of runs, including total emissions  & ASCII\\
 &  of different air pollutants per country& \\
%% eulmod.res & Mass budget for different compounds & ASCII\\
% INPUT.PARA & List of input parameters & ASCII\\\hline
% Remove.sh & Removes the links to the input data at the& ASCII\\
% & end of the run&\\\hline
Timing.out & Timing log file& ASCII \\
 
\hline
\end{tabular}
\end{center}

\label{Tab:outputs}
\end{table}


%ST: Paramer list

\newpage
\section{Output parameters netcdf files}\label{sec:OutputParam}

Parameters to be written out Base\_day.nc, Base\_month.nc and Base\_year.nc are defined in
My\_Derived\_ml.f90 and Derived\_ml.f90. In My\_Derived\_ml.f90, the
use can specify the output species (air concentrations, depositions,
column values), units and temporal resolution of the outputs (daily,
monthly, yearly).

The name of output parameter provides some information about data. The
names start with TYPE of the parameter, namely SURF (surface air
concentrations), DDEP (Dry deposition), WDEP (Wet deposition), COLUMN
(Vertically integrated parameters), Area (Surface area) etc.   

For suface air concentrations, the general name pattern is
SURF\_UNITS\_COMPONENT. Here, UNITS can e.g. be ug (\ug), ugS (\ugS),
ugN (\ugN), or ppb. The user can change units in My\_Derived\_ml.f90
(in array OutputConcs). Note that the components are classified either
as SPEC (species) or GROUP. The content of complex GROUP components
can be found in CM\_ChemGroups\_ml.f90.

For dry depositions, given per 1m$^{2}$ of specified landuse, the
names look like \\ DDEP\_COMPONENT\_m2LANDUSE, where LANDUSE can be
either a specific landuse type or a cell average. The units are \mgSm
or \mgNm. For wet depositions, the names are DDEP\_COMPONENT, and the
units are \mgSl or \mgNl.

For column integrated parameters, the names are
COLUMN\_COMPONENT\_NLAYERS, where NLAYERS is the number of layers
included in the integration. The units are molec/m$^2$, but can easily
be changed in My\_Derived\_ml.f90 and Derived\_ml.f90.

VG\_COMPONENT\_LANDUSE are the dry deposition velocities on various
landuse types, typically in cm/s.


Table \ref{tab:outpar} lists most of output parameters, providing
additional explanation to the complex components. For a complete suit
of currently selected output parameters, see provided output
NetCDF files, or My\_Derived\_ml.f90 module.

% $\nicefrac{\mu g}{m^{-3}}$

\newpage
\begin{center}
\begin{longtable}{lll}
\caption[List of output parameters]{List of output
  parameters (not complete).}\label{tab:outpar} \\

%\hspace{-1cm}
\hline
\textbf {Parameter name} &  {\textbf Short description} & {\textbf Comments}\\
\hline
\endfirsthead


\multicolumn{3}{c}%
	{\tablename\ \thetable\ -- \textit{Continued from previous
            page}} \\

\hline
\textbf{Parameter name} & \textbf{Short description} & \textbf{Comments}  \\
\hline
\endhead

\hline \multicolumn{3}{r}{\textit{Continued on next page}} \\ \hline
\endfoot

\hline \hline
\endlastfoot

%\hline
%{\bf Parameter name} &  {\bf Short description} & {\bf Comments}\\
%\hline
%\hline

%   & {\bf Air Concentrations}   & \\


    SURF\_ppb\_O3 & O$_{3}$ [ppb]&  \\
    SURF\_ugN\_NO & NO [\tugN]& Available also in ppb \\
%    SURF\_ppb\_NO & NO [ppb]&  \\
    SURF\_ugN\_NO2 & NO$_{2}$ [\tugN]& Available also in ppb \\
%    SURF\_ppb\_NO2 & NO$_{2}$ [ppb]&  \\
    SURF\_ugN\_HNO3 & HNO$_{3}$ [\tugN]&  Available also in ppb \\
    SURF\_ugN\_NH3 & NH$_{3}$ [\tugN]& Available also in ppb \\
    SURF\_ugS\_SO2 & SO$_{2}$ [\tugS]& Available also in ppb \\
    & & \\
    SURF\_ug\_SO4 & SO$_{4}^{2-}$ [\tug]&  \\
    SURF\_ug\_NO3\_F & NO$_{3}^{-}$ [\tug] fine aerosol & As ammonium nitrate \\
    SURF\_ug\_NO3\_C & NO$_{3}^{-}$ [\tug] coarse aerosol& Associated
    with sea salt and \\ & & mineral dust \\
    SURF\_ug\_TNO3 & NO$_{3}^{-}$ [\tug] total & Sum of fine and coarse nitrate \\
    SURF\_ug\_NH4\_F & NO$_{4}^{+}$ [\tug] fine aerosol& As ammonium
    sulphate and \\ & &  ammonium nitrate \\
    SURF\_ug\_SIA & SIA [\tug]& Secondary Inorganic Aerosol \\
    & & \\
    SURF\_ug\_ECFINE & EC fine [\tug]& Elemental carbon \\
    SURF\_ug\_ECCOARSE & EC coarse [\tug]& Elemental carbon \\
    SURF\_ug\_PART\_OM\_F & OM fine [\tug]& Organic Matter fine aerosol\\
    SURF\_ug\_OMCOARSE & OM coarse [\tug]& Organic Matter coarse
    aerosol\\
    & & \\
    SURF\_ug\_SEASALT\_F & Sea salt fine aerosol [\tug]&  \\
    SURF\_ug\_SEASALT\_C & Sea salt coarse aerosol [\tug]&  \\
    SURF\_ug\_SS & Sea salt  [\tug]& Sum of fine and coarse sea salt \\
    SURF\_ug\_DUST\_ROAD\_F & Road dust fine aerosol [\tug]&  \\
    SURF\_ug\_DUST\_ROAD\_C & Road dust coarse aerosol [\tug]&  \\
    SURF\_ug\_DUST\_WB\_F & Windblown dust fine  [\tug]&  \\
    SURF\_ug\_DUST\_WB\_C & Winblown dust coarse [\tug]&  \\
    SURF\_ug\_DUST\_SAH\_F & Saharan dust fine  [\tug]& From Boundary conditions \\
    SURF\_ug\_DUST\_SAH\_C & Saharan dust coarse [\tug]&From Boundary conditions \\
    SURF\_ug\_DUST\_NAT\_F & Natural dust fine  [\tug]& Windblown and Saharan \\
    SURF\_ug\_DUST\_NAT\_C & Natural dust coarse [\tug]& Windblown and Saharan \\
    SURF\_ug\_DUST & Mineral dust  [\tug]& From all sources \\
    & & \\
    SURF\_ug\_PM10 & PM$_{10}$ dry mass[\tug]&  \\
    SURF\_ug\_PM10\_rh50 & PM$_{10}$ wet [\tug]&PM$_{10}$ +\\ 
    & & particle water at Rh=50\% \\
    SURF\_ug\_PM25 & PM$_{2.5}$ dry [\tug]& Includes fine PM and  \\ & & 27\% of
    coarse $NO_{3}$  \\
    SURF\_ug\_PM25\_rh50 & PM$_{2.5}$ wet [\tug]& PM$_{2.5}$ + \\
    & & particle water at Rh=50\% \\
    SURF\_ug\_PM25X & PM$_{2.5}$ dry [\tug]& Includes fine PM and 27\% of
     \\ & & coarse $NO_{3}$, EC and OM \\
    SURF\_ug\_PM25X\_rh50 & PM$_{2.5}$ [\tug]& As PM$_{2.5}$X + \\ &&
    particle water at Rh=50\% \\

    SURF\_ug\_PMFINE & Fine PM [\tug]& Sum of all fine aerosols \\
    SURF\_ug\_PPM25 & Primary PPM$_{2.5}$ [\tug]& Anthropogenic emissions \\
    SURF\_ug\_PPM\_C & Primary coarse PM [\tug]& Anthropogenic
    emissions \\
    SURF\_ug\_PM25\_FIRE & PM$_{2.5}$ from forest fires [\tug]& Sum of BC, OC \\
    & & and rest PM$_{2.5}$ \\
etc.& &\\ \hline

& {\bf Dry Depositions}   & \\
\hline
    DDEP\_SOX\_m2Grid & Oxidized sulphur [\tmgSm]& For a grid cell\\
    & & landuse area weighted \\
    DDEP\_SOX\_m2Conif & Oxidized sulphur [\tmgSm]& To coniferous
    forest \\
    DDEP\_NOX\_m2Grid & Oxidized nitrogen [\tmgNm]& For a grid cell\\
    & & landuse area weighted \\
    DDEP\_NOX\_m2Decid & Oxidized nitrogen [\tmgNm]& To decideous
    forest \\
    DDEP\_RDN\_m2Grid & Reduced nitrogen [\tmgNm]& For a grid cell\\
    & & landuse area weighted \\
    DDEP\_RDN\_m2Seminat & Reduced nitrogen [\tmgNm]& To semi-natural \\
etc.& &\\ \hline
& {\bf Wet Depositions}   & \\
\hline
    WDEP\_PREC & Precipitation [mm]& \\
    WDEP\_SOX & Oxidized sulphur [\tmgSl]& \\
    WDEP\_SS  & Sea salt [\tmgl]& \\
etc.& &\\ \hline
& {\bf Others}   & \\
\hline
    AOD & Aerosol Optical Depth at 550nm & Experimental\\
    Area\_Crops\_Frac & Area fraction of crops & Available for several
    landuses\\
    VG\_NO3\_F\_Grid  & Dry deposition velocity of fine NO$_{3}^{-}$ &
    Grid cell average\\
etc.& &\\ \hline

& {\bf Meteorological parameters}   & \\
\hline
    USTAR\_GRID & U\* grid averaged & Available for several
    landuses\\
    T2m & Temperature at 2m [\degrees C] & \\
    rh2m  & Fractional relative humidity at 2m & \\
etc.& &\\ \hline
 
%\hline
%\end{tabular}
%\label{Tab:outpar}

\end{longtable}
\end{center}

\newpage
\section{ASCII outputs: sites and sondes}\label{sec:sitesonde}


Two main options are available for the output of ASCII files for comparison
with measurements or detailed model analysis. These are

\begin{description}
\item[sites]  

      output of surface concentrations for a set of specified
      measurement site locations.
\item[sondes] 

      output of concentrations for the vertical column above
     a set of specified locations.
\end{description}

Both sites and sondes are specified and handled in similar ways, in
the module {\bf Sites\_ml.f90}, so we treat them both together below.
Locations are specified in the input files ``sites.dat'' and ``sondes.dat''. 
The files start with a description of its content
followed by a list of the stations. For example, a sondes.dat input file
may look like this:

\begin{small}
\begin{quote}
\begin{verbatim}
# "Sondes: names, locations, elevations"
# "Area: EMEP-Europe"
# "ix: x coordinate"
# "iy: y coordinate"
# "lev: vertical coordinate (20=ground)"
: Units index
: Coords LatLong
: VertCoords EMEPsigma
: DomainName NA
#
name lat long lev #HEADERS
-    deg deg level #SKIP
#DATA:
Uccle             50.80    4.35  20  ! comment
Lerwick           60.13   -1.18  20  ! comment
Sodankyla         67.39   26.65  20  ! comment
Ny_Alesund        78.93   11.88  20  ! comment
Hohenpeissenberg  47.80   11.02  20  ! comment
\end{verbatim}

\end{quote}
\end{small}


The first line in each file is a header with file content.
Then, the contents are described in more detail. Text strings after
\# are just clarifying comments. 'Area', e.g., is the domain to which
the stations belong, e.g. 'Northern Hemisphere'.

Text after ':' is read in by the model:\newline
- Units: either 'deg' (degrees) or 'index' (model grid indices)\newline
- Coords: either 'LatLong' (latitudes/longitudes) or 'ModelCoords'
(indices of the grid box in which the station is located)\newline
- VertCoords: vertical coordinate system that is used in the model (usually
'EMEPsigma')\newline


Both sites.dat and sondes.dat files are optional, but recommended. 
The species and meteorological data requested for site and sone output
 are specified in {\bf My\_Outputs.f90} by the use of arrays. Only a 
few met fields are defined so far but more can be added into 
{\bf Sites\_ml.f90} as required. 

The output files sites\_2010.csv and sondes\_2010.csv are comma separated files 
that can be read by excel. If you include the whole year, or the 31. December, sites\_2011.csv 
and sondes\_2011.csv are also incued in the output.
% The outputs consist of a header giving the number of sites requested, 
% species/met data requested, and then the actual values for each 
% variable for each staion. For the sonde data values are given for 
% all 20 levels, starting with the ground-level values. These files
% are best read using some helper programmes, which we provide in
% the "Tools" directory:

% \begin{description}
% \item[Rd\_sites.f90] - reads the sites outout file for one month, and allows the
% user to select site and component. Produces several outputs (SITES.hrly etc.), including
% hourly values, daily mean and max. The code is quite short and easy to modify
% \item[Rd\_sondes.f90] - similar to Rd\_sites, but for the sondes output.
% \end{description}
% 
% Simply compile and run the executable to obtain usage instructions.


% \newpage

% \section{NetCDF outputs, Derived fields}
% 
% %{\bf modules My\_Derived\_ml, Derived\_ml }\\
% 
% % My\_Derived\_ml:\\
% % \begin{itemize}
% %   \item Constructs arrays wanted\_deriv2d, wanted\_deriv3d, listing
% %    the outputs which the user wants in netcdf fields.
% % \end{itemize}
% % 
% % Derived\_ml:\\
% % \begin{itemize}
% %   \item Contains definitions of possible output fields
% %   \item Allocates data arrays d\_2d and d\_3d for those
% %     fields which were requested in My\_Derived\_ml	
% % \end{itemize}
% 
% The netCDF output files can contain both 2D and 3D data fields. The module 
% { \bf Derived\_ml.f90 } contains an extensive set of predefined outputs, and
% others are defined as required in { \bf My\_Derived\_ml.f90}. 
% The user may choose which of these possible data fields should be written 
% into the output files by modifying the module called { \bf My\_Derived\_ml.f90}.\\
% 
% Most data fields are defined by the array OutputConcs. A defined output 
% field for ozone in ppb is for example given  as:
% \begin{quote}
% \begin{verbatim}
% 
%  typ_s5i("O3        ", "ppb", D2,"AIR_CONCS", SPEC, D),& 
% 
% \end{verbatim}
% \end{quote}
% This will give a 'SURF\_ppb\_O3' 2-dimensional surface field in the output netCDF file for 
% the fullrun, monthly and daily output. 
% Setting the dimensions to D3 will give an additional 3-dimensional field. 
% 
% 
% Other arrays where output fields are given are D2\_SR, D2\_EXTRA,\
%  VEGO3\_WANTED and 
% WDEP\_WANTED. Other fields are also defined in { \bf My\_Derived\_ml.f90 } e.g. for dry 
% depositions, emissions, canopy etc. \\
% 
% 
% 
% Note that by default the maximum field of 2- and 3-dimensional output field are 
% restricted to 200 and 5 respectively. These can be changed with MAX\_NUM\_DERIV2D and MAX\_NUM\_DERIV3D 
% in { \bf My\_Derived\_ml.f90 }.
% 
% 
% The output frequency is set in { \bf Derived\_ml.f90 }, for example for the surface fields:
% \begin{quote}
% \begin{verbatim}
% 
%  call AddNewDeriv( "PSURF ","PSURF",  "SURF","-",   "hPa", &
%                -99,  -99,  F,  1.0,  T,   IOU_DAY )
% 
% \end{verbatim}
% \end{quote}
%  Where the IOU\_DAY setting gives the field in the fullrun.nc, month.nc in addition to the day.nc 
% model output files. IOU\_MON gives the field in the fullrun.nc and the month.nc file. 
% 
% 

