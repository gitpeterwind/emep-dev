\chapter{Model architecture and variable names}

The picture below ilustrates the vertical structure of the model. The model 
is using sigma coordinates. There are 20 vertical layers from 100 hPa to 
the surface in the model. z\_bnd denotes the devision between the layers 
(z\_bnd(1) is the height of the top of the model and z\_bnd(21) the surface.
z\_mid(1) is the height of the midpoint of the top layer and z\_mid(20) the 
height of the mid point of the lowest model layer.

\begin{picture}(400,300)
\linethickness{0.5mm}\scriptsize
\multiput(0,25)(0,50){6}{\line(1,0){250}}
\linethickness{0.25mm}\scriptsize
\multiput(0,50)(0,50){5}{\line(1,0){100}}
\put(140,10){\makebox(0,0){\large Surface}}
\put(320,20){\makebox(0,0)[b]{{\large z\_bnd(KMAX\_BND) }}}
\put(330,70){\makebox(0,0)[b]{{\large z\_bnd(KMAX\_BND - 1) }}}
\put(330,120){\makebox(0,0)[b]{{\large z\_bnd(KMAX\_BND - 2) }}}
\put(330,170){\makebox(0,0)[b]{{\large z\_bnd(KMAX\_BND - 3) }}}

\put(190,195){\makebox(0,0)[b]{{\large z\_mid(KMAX\_MID-3) }}}
\put(190,145){\makebox(0,0)[b]{{\large z\_mid(KMAX\_MID-2) }}}
\put(190,95){\makebox(0,0)[b]{{\large z\_mid(KMAX\_MID-1) }}}
\put(180,45){\makebox(0,0)[b]{{\large z\_mid(KMAX\_MID) }}}
\end{picture}
\vskip 5 mm

In the horizontal the model use a 50x50 km polar stereographic mapping, 
true at 60$^{\circ}$N. 
The horizontal topography is ilustrated in the picture below by the 
south-west corner of the 
model domain. The horizontal vinds (u and v) are given on a staggered grid 
(this is also the case with the vertical vind component sdot). All other 
variables (represented as q in the picture) are given in the centre of the 
grid.

\begin{picture}(400,250)
\linethickness{0.5mm}\scriptsize
%draw the lines
\multiput(30,30)(0,75){3}{\line(1,0){300}}
\multiput(30,30)(75,0){5}{\line(0,1){150}}
%make the dots
\put(67,67){\circle*{2}}
\put(142,67){\circle*{2}}
\put(217,67){\circle*{2}}
\put(292,67){\circle*{2}}

\put(67,142){\circle*{2}}
\put(142,142){\circle*{2}}
\put(217,142){\circle*{2}}
\put(292,142){\circle*{2}}
%write u,v
\put(30,10){\makebox(0,0)[b]{{\large u,v$_{(0,0)}$ }}}
\put(105,10){\makebox(0,0)[b]{{\large u,v$_{(1,0)}$ }}}
\put(180,10){\makebox(0,0)[b]{{\large u,v$_{(2,0)}$ }}}
\put(255,10){\makebox(0,0)[b]{{\large u,v$_{(3,0)}$ }}}

\put(0,105){\makebox(0,0)[b]{{\large u,v$_{(0,1)}$ }}}
\put(0,180){\makebox(0,0)[b]{{\large u,v$_{(0,2)}$ }}}


\put(67,50){\makebox(0,0)[b]{{\large q$_{(1,1)}$ }}}
\put(142,50){\makebox(0,0)[b]{{\large q$_{(2,1)}$ }}}
\put(217,50){\makebox(0,0)[b]{{\large q$_{(3,1)}$ }}}
\put(292,50){\makebox(0,0)[b]{{\large q$_{(4,1)}$ }}}

\put(67,125){\makebox(0,0)[b]{{\large q$_{(1,2)}$ }}}
\put(142,125){\makebox(0,0)[b]{{\large q$_{(2,2)}$ }}}
\put(217,125){\makebox(0,0)[b]{{\large q$_{(3,2)}$ }}}
\put(292,125){\makebox(0,0)[b]{{\large q$_{(4,2)}$ }}}

\end{picture}


In the tables below the most important variable names used in the Unified 
model will be listed. For some parameters a reference will be given to other 
sections of the documentation in the ``ref.'' column in the tables.


\begin{table} [h]
\begin{center}
\caption{Meteorological input data. The data are in general intentaneous 
values unless listed as 3 h avg. in the dimension column in wich case the 
variable is givea as a 3 hour average.}
\label {table:variable_inmet}
\begin{tabular} {l|l|l|l|l}
Variable name & description             &dimension& unit       & ref.   \\
\hline
u             & vind in x - dirrection  & 3 d   & m s$^{-1}$   &        \\
v             & vind in y - dirrection  & 3 d   & m s$^{-1}$   &        \\
sdot          & vertical vind component & 3 d   & Pa s$^{-1}$  &        \\
th            & potential temperature   & 3 d   & Deg. K       &        \\
q             & specific humidity       &3 d    & mass ratio &        \\
cw            & cloud water            & 3 d, 3 h avg.& kg$_w$/kg$_{air}$  & \\
cc3d          & cloud cover 3 h average & 3 d, 3 h avg.& percent   &        \\
pr            & 3 h precip. summed from abv.& 3 d, 3 h avg. & kg m$^{-2}$ &  \\
ps            & surface pressure        & 2 d   & Pa           &        \\
th2m          & teperature at 2 m       & 2 d   & Deg. K       &        \\
fh            & surface flux sens. heat & 2 d   & W m$^2$      &        \\
fm            & surface stress          & 2 d   & N m$^2$      &        \\ 

\end {tabular}
\end {center}
\end {table}


\begin{table} [h]
\begin{center}
\caption{Derived meteorological data.The data are in general intentaneous 
values unless listed as 3 h avg. in the dimension column in wich case the 
variable is givea as a 3 hour average.}
\label {table:variable_derivmet}
\begin{tabular} {l|l|l|l|l}
Variable name & description             &dimension& unit       & ref.   \\
\hline
roa           & density                 & 3 d     & kg m$^{-3}$&        \\
cc3dmax       & max. cloud cover above  & 3 d, 3 h avg. & percent  &      \\
p1            & pressure at sigma\_bnd   & 3 d     & Pa         &        \\
z\_bnd         & height at sigma\_bnd     & 3 d     & m          &        \\
z\_mid         & height at sigma\_mid     & 3 d     & m          &        \\
skh           & exhange coefficient     & 3 d     & ??         &        \\
\end {tabular}
\end {center}
\end {table}

