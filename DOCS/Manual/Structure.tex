\chapter{Model structure}

The model is composed of a large number of modules together with a few
standalone subroutines. The intention is that \emph{only}
modules which are prefixed `{\bf My\_}'  should change for different
model versions. Thus, changing from ozone to acidification models
should consist of changing only these modules. It is envisaged that
in future such My\_ files can be collected into separate directories
and loaded into the main model direcory when required.

Files which do not begin with `My\_' should not need to be changed
at all for different model versions. Think very carefully before
making any changes here - there will hopefully be an option to
do what you want  in the My\_ files!! If not, we need to create
such an option, or persuade you to make do with another method ;-)


Currently, two sub-directories, {\bf ZD\_OZONE} and {\bf ZD\_ACID}
are used to store the various `My' files. SImply copy these
into the main directory befor compilation to get the version required.

\begin{small}
\begin{table}
\caption{The Unified model files}
\begin{tabular}{|llll|} \hline
Unimod.f90  &                      &                    &  \\
\hline
BoundaryConditions\_ml.f90 & My\_BoundConditions\_ml.f90 & UiO\_ml.f90  &  \\
\hline
Chem\_ml.f90 &  My\_Chem\_ml& My\_Reactions.in  &  \\
Aqueous\_ml.f90 &  &   &  \\
Ammonium\_ml.f90 &  &   &  \\
DefPhotolysis\_ml.f90 &  &   &  \\
DryDep\_ml.f90     & My\_DryDep\_ml &   &  \\
WetDep\_ml.f90     & My\_WetDep\_ml &   &  \\
OrganicAerosol\_ml.f90 &  &   &  \\
Runchem\_ml.f90 &  &   &  \\
Solver.f90 &  &   &  \\
\hline
EmisGet\_ml.f90 & My\_Emis\_ml.f90 & EmisDef\_ml.f90  &  \\
Emissions\_ml.f90 &  &   &  \\
Biogenics\_ml.f90 &  &   &  \\
AirEmis\_ml.f90 &  &   &  \\
Timefactors\_ml.f90 &  &   &  \\
\hline
ModelConstants\_ml.f90 &  &   &  \\
PhysicalConstants\_ml.f90 &  &   &  \\
\hline
Io\_ml.f90 &  &   &  \\
NetCDF\_ml.f &  Derived\_ml.f90 & My\_Derived\_ml.f90  &   \\
Output\_hourly.f &  My\_Outputs\_ml.f90 &   &  \\
Out\_restri\_ml.f90 &  &   &  \\
outchem\_restri.f &  &   &  \\
Sites\_ml.f90 &  &   &  \\
\hline
%gc\_com.F &  & Par\_ml   &  \\
%getflti2.F &  &   &  \\
%parinit.f &  &   &  \\
%putflti2.F &  &   &  \\
%put\_restri\_i2.F &  &   &  \\
%\hline
Advection\_ml.f90 &  &   &  \\
Bud\_ml.f90 &  &   &  \\
Country\_ml.f90 &  &   &  \\
Dates\_ml.f90 &  &   &  \\
Functions\_ml.f90 &  &   &  \\
GridValues\_ml.f90 &  &   &  \\
Met\_ml.f90 &  &   &  \\
Polinat\_ml.f90 &  &   &  \\
Setup\_1d\_ml.f90 &  &   &  \\
Setup\_1dfields\_ml.f90 &  &   &  \\
Tabulations\_ml.f90 &  &   &  \\
\hline
gc\_com.F &  & Par\_ml   &  \\
getflti2.F &  &   &  \\
parinit.f &  &   &  \\
putflti2.F &  &   &  \\
put\_restri\_i2.F &  &   &  \\
\hline
global2local.f &  &   &  \\
local2global.f &  &   &  \\
phyche.f &  &   &  \\
zenit\_angle.f &  &   &  \\
\hline
\end{tabular}
\end{table}
\end{small}



\chapter{The Code}



\begin{itemize}
\item
95\% Fortran 90/95

\item
Preferably in F
   \begin{itemize}
      \item
        A sub-set of F95
   \end{itemize}

\item
   Heavy use of 'modern' fortran and modules
   \begin{itemize}
      \item
        e.g. No common blocks
      \item
        intent attributes  in subroutine arguments
   \end{itemize}
\end{itemize}

\section*{A simple module}

\begin{verbatim}
      module PhysicalConstants_ml
!----------------------------------------------------------------------------
!  Defines Physical constants
!----------------------------------------------------------------------------
implicit none
private

  real , public, parameter ::         &
    AVOG   = 6.023e23                 & ! Avogadros number
  , ATWAIR = 28.964                   & ! mol wt of air, g/mol
  , RGAS_ATML = 0.08205               & ! Molar Gas constant (atm M-1 K-1)
  , RGAS_KG   = 287.0                 & ! Molar Gas constant (J K-1 kg-1)
  , RGAS_J    = 8.314                   ! Molar Gas constant (J mol-1 K-1)

  real, public, parameter  ::    &
       GRAV    = 9.807           &   ! Gravity, m s-2
  ....
    ,  FREEPATH  = 6.5e-8        &   ! Mean Free Path of air [m]
  ....

\end{verbatim}



%\newpage
To use, e.g.  AVOG from another module, one usually uses:

\begin{verbatim}

module Demo1
  use PhysicalConstants_ml, only : AVOG
 ...
end module Demo1

\end{verbatim}

More complex modules mix subroutines and data.


A typical EMEP code module looks like
\begin{verbatim}

module Demo2
  use PhysicalConstants_ml, only : AVOG
  use Rsurface_ml,  only : Rsurface, Gsto, Rb
 ...

  !/ Subroutines:

  public :: demo_sub
  private :: Initialise
  private :: Tabulate


  logical, save, private :: my_first_call = .true.

 contains
   subroutine demo_sub(a,b,c)
     real, intent(in) :: a, b
     real, intent(out) :: c

     if ( my_first_call ) then
           call Initialise
           call Tabulate
           my_first_call = .false.
     end if

     call Rsurface(....)

   end subroutine demo_sun

   subroutine Initialise
      bla bla 
   end subroutine Initialise

   subroutine Tabulate
      bla bla 
   end subroutine Tabulate

end module Demo1

\end{verbatim}
